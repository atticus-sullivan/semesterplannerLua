% \iffalse meta-comment
% vim: textwidth=75
% vim: set tabstop=4 shiftwidth=4 expandtab
%<*internal>
\iffalse
%</internal>
%<*readme>
|
-------------------:| -----------------------------------------------------
semesterplanner-lua:| Semesterplanner package in lua with tikz only
             Author:| Lukas Heindl
             GitLab:| https://gitlab.com/AtticusSullivan/semesterplanner-lua
            License:| Released under the LaTeX Project Public License v1.3c or later
                See:| http://www.latex-project.org/lppl.txt


Short description:
Some text about the package: probably the same as the abstract.
%</readme>
%<*internal>
\fi
\def\nameofplainTeX{plain}
\ifx\fmtname\nameofplainTeX\else
  \expandafter\begingroup
\fi
%</internal>
%<*install>
\input docstrip.tex
\keepsilent
\askforoverwritefalse
\preamble
-------------------:| -----------------------------------------------------
semesterplanner-lua:| Semesterplanner package in lua with tikz only
             Author:| Lukas Heindl
             GitLab:| https://gitlab.com/AtticusSullivan/semesterplanner-lua
            License:| Released under the LaTeX Project Public License v1.3c or later
                See:| http://www.latex-project.org/lppl.txt

\endpreamble
\postamble

Copyright (C) 2021 by Lukas Heindl

This work may be distributed and/or modified under the
conditions of the LaTeX Project Public License (LPPL), either
version 1.3c of this license or (at your option) any later
version.  The latest version of this license is in the file:

http://www.latex-project.org/lppl.txt

This work is "maintained" (as per LPPL maintenance status) by
(not set).

This work consists of the file semesterplanner-lua.dtx and a Makefile.
Running "make" generates the derived files README, semesterplanner-lua.pdf and semesterplanner-lua.sty.
Running "make inst" installs the files in the user's TeX tree.
Running "make install" installs the files in the local TeX tree.

\endpostamble

\usedir{tex/latex/semesterplanner-lua}
\generate{
  \file{\jobname.sty}{\from{\jobname.dtx}{package}}
}
\usedir{tex/latex/semesterplanner-lua}
\generate{
  \nopreamble\nopostamble
  \file{\jobname.lua}{\from{\jobname.dtx}{luaMain}}
}
%</install>
%<install>\endbatchfile
%<*internal>
\usedir{source/latex/semesterplanner-lua}
\generate{
  \file{\jobname.ins}{\from{\jobname.dtx}{install}}
}
\nopreamble\nopostamble
\usedir{doc/latex/semesterplanner-lua}
\generate{
  \file{README.txt}{\from{\jobname.dtx}{readme}}
}
\ifx\fmtname\nameofplainTeX
  \expandafter\endbatchfile
\else
  \expandafter\endgroup
\fi
%</internal>
% \fi
%
% \iffalse
%<*driver>
\ProvidesFile{semesterplanner-lua.dtx}
%</driver>
%<*driver>
\documentclass{ltxdoc}
\usepackage[a4paper,margin=25mm,left=50mm,nohead]{geometry}
\usepackage[numbered]{hypdoc}
\usepackage{\jobname}
\EnableCrossrefs
\CodelineIndex
\RecordChanges
\begin{document}
  \DocInput{\jobname.dtx}
\end{document}
%</driver>
%<package>\NeedsTeXFormat{LaTeX2e}[1999/12/01]
%<package>\ProvidesPackage{semesterplanner-lua}
%<*package>
    [2021/10/07 v0.2 Semesterplanner package in lua with tikz only]

\RequirePackage{tikz}
\RequirePackage{pgfkeys}
\RequirePackage{xcolor}
\RequirePackage{fontawesome}
%</package>
% \fi
%
% \GetFileInfo{\jobname.dtx}
% \DoNotIndex{\newcommand,\newenvironment,\\,\begin,\end,\def,\definecolor,\directlua,\endinput,\faBullhorn,\faCamera,\faFlag,\faQuestion,\faTimesCircle,\faWarning,\faWindows,\faYoutubePlay,\node,\pgfkeys,\pgfkeysvalueof,\protected,\raggedright,\textbf,\textcolor,\textwidth,\unexpanded,\"}
%
%\title{\textsf{semesterplanner-lua} --- Semesterplanner package in lua
%with tikz only\thanks{This file describes version \fileversion, last
%revised \filedate.}
%}
%\author{Lukas Heindl\\GitLab:
%\url{https://gitlab.com/AtticusSullivan/semesterplanner-lua}}
%\date{Released \filedate}
%
%
%\maketitle
%
%\changes{v1.00}{2021/10/07}{First public release}
%
% \begin{abstract}
% ==== Put abstract text here. ====
% \end{abstract}
%
% \tableofcontents
%
% \section{Usage}
%
% ==== Put descriptive text here. ====
%
% \DescribeMacro{\dummyMacro}
% This macro does nothing.\index{doing nothing|usage} It is merely an
% example.  If this were a real macro, you would put a paragraph here
% describing what the macro is supposed to do, what its mandatory and
% optional arguments are, and so forth.
%
% \DescribeEnv{dummyEnv}
% This environment does nothing.  It is merely an example.
% If this were a real environment, you would put a paragraph here
% describing what the environment is supposed to do, what its
% mandatory and optional arguments are, and so forth.
%
% \begin{timetable}[days={Mon,Thue,Wend,Thur,Fri}, start
% time/.evaluated={11*60}, end time/.evaluated={15*60}]
%   \lecture{TestingLectureLongOne}{Heindl}{RN1}{W}{12:30-13:30}{}{}
%   \lecture{TestingLectureLongOne}{Heindl}{\zoom}{T}{12:30-13:30}{\phigh}{}
% \end{timetable}
%
% Special about the |length| argument is that the height of the column
% headers (namely the weekdaynames) isn't counted to the length you can
% specify.
%
% Same goes for the width regarding the labels containing the time on the
% right. Since in this case any tex lenght is allowed, you can simply try
% to subtract the length of the clock label using something like
% |\settowidth{\length}{12:30}| to set a length to the length of a clock
% label and then subtract this from the length you want to specify.
%
% Hint: The content of the environment isn't processed by thispackage, only
% the event commands are relevant. All other contents are set immediately
% before the timetable. Therefore, if you wan to add e.g. a
% |\hspace*{10cm}| before the timetable, the last line of the env would be
% the place to do so (there musn't be an empty line below since otherwise a
%  new paragraph is started).
%
%
%\StopEventually{
%  \PrintChanges
%  \PrintIndex
%}
%
% \section{Implementation}
% This package uses |semesterplanner-lua| as prefix/directory where
% possible. Since this is not possible for latex macro names, in this
% occasions |semesterplannerLua@| is used as prefix.
% \subsection{semesterplanner-lua.sty}
% \subsubsection{Global Stuff}
%    \begin{macrocode}
%<*package>
%    \end{macrocode}
% Define some colors for the course types (can be globally overwritten)
%    \begin{macrocode}
\definecolor{seminar}{rgb}{1.0, 0.8, 0.0}
\definecolor{lecture}{rgb}{0.2, 0.7, 1.0}
\definecolor{tutorial}{rgb}{0.0, 0.8, 0.0}
\definecolor{meeting}{rgb}{0.8, 0.0, 0.0}
\definecolor{officehour}{rgb}{0.0, 0.4, 0.6}
\definecolor{DodgerBlue}{HTML}{1E90FF}
%    \end{macrocode}

% Load the lua module
%    \begin{macrocode}
\directlua{sp = require("semesterplanner-lua.lua")}
%    \end{macrocode}
%
% \subsubsection{Local Stuff (timetable-env local)}
% \begin{environment}{timetable}
% This is the environment doing all the stuff. To gate the positions where
% the corresponding macros can be used (and in terms of pgfkeys for reasons
% of default values) all the macros used are put into the environment.
%    \begin{macrocode}
\newenvironment{timetable}[1][]{
%    \end{macrocode}
% \begin{macro}{\semesterplannerLua@encircle}
% This macro puts a circle arround its argument for better readability. In
% this package this is used for the fontawesome symbols.
%    \begin{macrocode}
    \newcommand*{\semesterplannerLua@encircle}[1]{
        \begin{minipage}[b][1em][c]{1.5em}
            \begin{tikzpicture}
                \node[fill,circle,inner sep=1pt, color = white] {##1};
            \end{tikzpicture}   
        \end{minipage}
    }
%    \end{macrocode}
% \end{macro}
%
% Set all the pgfkeys required for the arguments. To achieve that the
% defaults are restored every time the environment is used, this is
% inside the environment definition. This of course disables all
% possibilities of setting a global default but enables setting local
% defaults for the events
%    \begin{macrocode}
    \pgfkeys{
%    \end{macrocode}
% |/semesterplanner-lua| will be the pgf-path used for this package
% Set the environment arguments arguments. |days|, |width| and
% |height| are used later in drawing. |start time| and |end time| are
% important for collecting the events as well.
% \begin{description}
%   \item[|days|] is a list of
%       strings representing the header names for the day columns in the
%       timetable (adding Sat and Sun (additional entries) will result in
%       two more columns.
%   \item[|length|] is the vertical length of the
%       timetable (not including the clock labels on the side)
%       measured in cm (in future versions this may become measured in
%       pts for better interaction with the LaTeX lengths.
%   \item[|width|] is
%       the horizontal width of the timetable (not including the column
%       headers on the top) this can be a latex length string or
%       |\textwidth| as well.
%   \item[|start time|] can be used to set a fixed time where the timetable
%       starts (otherwise this is calculated from the entries) to enable
%       this behaviour this key has to be set to |HH*60 + MM| (easy way
%       is by using |start time/.evaluated={HH*60+MM}|)
%   \item[|end time|] equivalent to |start time|
% \end{description}
%    \begin{macrocode}
        /semesterplanner-lua/.cd,
        days/.initial={Mon,Thue,Wend,Thur,Fri},
        days/.default={Mon,Thue,Wend,Thur,Fri},
        %
        start time/.initial=,
        start time/.default=,
        end time/.initial=,
        end time/.default=,
        %
        width/.initial=\textwidth,
        width/.default=\textwidth,
        length/.initial=10,
        length/.default=10,
        %
%    \end{macrocode}
% |/semesterplanner-lua/event| is the path where the keys relevant
% for the event macro resides
% \begin{description}
%   \item[|content|] is the content of the event (is passed on without any
%           formatting). Since this is passed to lua without modification
%           its value must be an unexpanded string (lua will simply print
%           it so the eventually the string will be evaluated)
%   \item[|time|] is a |HH:MM-HH:MM| string representing start- and
%       end-time of the event
%   \item[|day|] is either |M|,|T|,|W|,|Th| or |F| specifying the day on
%       which the event takes place
%   \item[|tikz|] this key allows the user to manually pass options to the
%       node created for this event
%   \item[|scale width|] allows to scale the width of the event to be able
%       to draw overlapping events besides each other. Will usually be a
%       value between |0| and |1|.
%   \item[|offset|] same goal like |scale width| but shifts the event node
%       by the given value to the right. (Given as value between |0| and
%       |1| indicating how many columns the event should be shifted)
% \end{description}
%    \begin{macrocode}
        event/.cd,
        % event arguments
        content/.initial=,
        content/.default=,
        %
        time/.initial=,
        time/.default=,
        day/.initial=,
        day/.default=,
        %
        tikz/.initial=,
        tikz/.default=,
        scale width/.initial=1,
        scale width/.default=1,
        offset/.initial=0,
        offset/.default=0,
    }
%    \end{macrocode}

%   Commands for symbols of priority
%   \begin{macro}{\pmandatory}
%    \begin{macrocode}
    \protected\def\pmandatory{\semesterplannerLua@encircle{\textcolor{red}{\faWarning}}}
%    \end{macrocode}
%   \end{macro}
%   \begin{macro}{\phigh}
%    \begin{macrocode}
    \protected\def\phigh{\semesterplannerLua@encircle{\textcolor{red}{\faFlag}}}
%    \end{macrocode}
%   \end{macro}
%   \begin{macro}{\pmid}
%    \begin{macrocode}
    \protected\def\pmid{\semesterplannerLua@encircle{\textcolor{yellow}{\faFlag}}}
%    \end{macrocode}
%   \end{macro}
%   \begin{macro}{\plow}
%    \begin{macrocode}
    \protected\def\plow{\semesterplannerLua@encircle{\textcolor{green}{\faFlag}}}
%    \end{macrocode}
%   \end{macro}
%   \begin{macro}{\pnone}
%    \begin{macrocode}
    \protected\def\pnone{\semesterplannerLua@encircle{\textcolor{gray}{\faTimesCircle}}}
%    \end{macrocode}
%   \end{macro}

%   Commands for online platforms.
%   \begin{macro}{\teams}
%    \begin{macrocode}
    \protected\def\teams{\semesterplannerLua@encircle{\textcolor{DodgerBlue}{\faWindows}}}
%    \end{macrocode}
%   \end{macro}
%   \begin{macro}{\zoom}
%    \begin{macrocode}
    \protected\def\zoom{\semesterplannerLua@encircle{\textcolor{DodgerBlue}{\faCamera}}}
%    \end{macrocode}
%   \end{macro}
%   \begin{macro}{\youtube}
%    \begin{macrocode}
    \protected\def\youtube{\semesterplannerLua@encircle{\textcolor{red}{\faYoutubePlay}}}
%    \end{macrocode}
%   \end{macro}

%   Command for "To be determined" and "To be Announced"
%   \begin{macro}{\tbd}
%    \begin{macrocode}
    \protected\def\tbd{\faQuestion}
%    \end{macrocode}
%   \end{macro}
%   \begin{macro}{\tba}
%    \begin{macrocode}
    \protected\def\tba{\faBullhorn}
%    \end{macrocode}
%   \end{macro}

%   Read the argumens given by the user after restoring the defaults
%   (Restoring currently makes no sense, since they are created a few
%   lines above anyways, but creation might be moved outside the
%   environment some day.\\
%   Afterwards the lua module is beeing initialized (erase data from
%   possible previous runs.
%    \begin{macrocode}
    \pgfkeys{/semesterplanner-lua/.cd, days,length,width,start time,end time, #1}
    \directlua{sp.init(
        "\pgfkeysvalueof{/semesterplanner-lua/days}",
        "\pgfkeysvalueof{/semesterplanner-lua/start time}",
        "\pgfkeysvalueof{/semesterplanner-lua/end time}")}
%    \end{macrocode}
%   \begin{macro}{\semesterplanner@event}
% Is used to pass the event to the lua engine which in turn will collect
% the event to draw it in the end. For that the arguments given are parsed
% after restoring the pgf keys to their default values. The optional
% argument herby is a sequence of pgf keys, the second argument is
% a string representing the content (this MUST be unexpanded since
% this is passed to lua which in turn will pass it unmodified back)
%    \begin{macrocode}
    \newcommand{\semesterplannerLua@event}[2][]{
        \pgfkeys{/semesterplanner-lua/event/.cd,content,time,day,tikz,scale width,
        offset, ##1, content=##2}
        \directlua{
            sp.addEvent{
                time="\pgfkeysvalueof{/semesterplanner-lua/event/time}",
                day="\pgfkeysvalueof{/semesterplanner-lua/event/day}",
                tikz=[[\pgfkeysvalueof{/semesterplanner-lua/event/tikz}]],
                content=[[\pgfkeysvalueof{/semesterplanner-lua/event/content}]],
                offset=\pgfkeysvalueof{/semesterplanner-lua/event/offset},
                scale_width=\pgfkeysvalueof{/semesterplanner-lua/event/scale width},
            }
        }
    }
%    \end{macrocode}
%   \end{macro}
%   \begin{macro}{\semesterplannerLua@formattedEvent}
% Simply a layer above |\semesterplannerLua@event| which formats the content
% before passing it on. This formatting is thought to be a good formatting
% for lecture-like entries and is heavily stolen from
% \footnote{\url{https://github.com/nlschn/semesterplanner/}}
% Takes a number of arguments:
% \begin{enumerate}
%   \item title of the event
%   \item name of the speaker/lecturer
%   \item location (e.g. roomnumber)
%   \item day on which the event takes place (for valid values see the
%       |day| pgf key above)
%   \item time (for valid values / formatting see the |time| pgf key above)
%   \item priority of the event (no special formatting needed, consider
%       using one of |\phigh|, \ldots
%   \item event code. This is passed to event-pgf unmodified and can
%       overwrite any of the above keys. To add some arguments to tikz
%       simply use |tikz/.append={draw=green}|
%   \item background color of the event
%   \item text color of the content
% \end{enumerate}
%    \begin{macrocode}
    \def\semesterplannerLua@formattedEvent##1##2##3##4##5##6##7##8##9{
        \semesterplannerLua@event[time=##5, day=##4, tikz={fill=##8}, ##7]
        {
            \unexpanded{
                \textcolor{##9}{
                    \textbf{##1}\\[.2em]
                    \raggedright{##2}\\[0.5em]\raggedright{##6}\raggedright{##3}
                }
            }
        }
    }
%    \end{macrocode}
%   \end{macro}
% Short-hand macros for different events using the corresponding
% background color
%
%   \begin{macro}{\lecture}
%    \begin{macrocode}
    \def\lecture##1##2##3##4##5##6##7{
        \semesterplannerLua@formattedEvent{##1}{##2}{##3}{##4}{##5}{##6}{##7}{lecture}{white}
    }
%    \end{macrocode}
%   \end{macro}
%   \begin{macro}{\seminar}
%    \begin{macrocode}
    \def\seminar##1##2##3##4##5##6##7{ %##1=title, ##2=speaker, ##3=location, ##4=day, ##5=time, ##6=priority, ##7=event-code (tikz can eb set this way too but you must use append)
        \semesterplannerLua@formattedEvent{##1}{##2}{##3}{##4}{##5}{##6}{##7}{seminar}{white}
    }
%    \end{macrocode}
%   \end{macro}
%   \begin{macro}{\tutorial}
%    \begin{macrocode}
    \def\tutorial##1##2##3##4##5##6##7{ %##1=title, ##2=speaker, ##3=location, ##4=day, ##5=time, ##6=priority, ##7=event-code (tikz can eb set this way too but you must use append)
        \semesterplannerLua@formattedEvent{##1}{##2}{##3}{##4}{##5}{##6}{##7}{tutorial}{white}
    }
%    \end{macrocode}
%   \end{macro}
%   \begin{macro}{\meeting}
%    \begin{macrocode}
    \def\meeting##1##2##3##4##5##6##7{ %##1=title, ##2=speaker, ##3=location, ##4=day, ##5=time, ##6=priority, ##7=event-code (tikz can eb set this way too but you must use append)
        \semesterplannerLua@formattedEvent{##1}{##2}{##3}{##4}{##5}{##6}{##7}{meeting}{white}
    }
%    \end{macrocode}
%   \end{macro}
%   \begin{macro}{\officehour}
%    \begin{macrocode}
    \def\officehour##1##2##3##4##5##6##7{ %##1=title, ##2=speaker, ##3=location, ##4=day, ##5=time, ##6=priority, ##7=event-code (tikz can eb set this way too but you must use append)
        \semesterplannerLua@formattedEvent{##1}{##2}{##3}{##4}{##5}{##6}{##7}{officehour}{white}
    }
%    \end{macrocode}
%   \end{macro}
%    \begin{macrocode}
}{
%    \end{macrocode}
% At the end of the environment after all events have been collected,
% generate and output the tikz code needed to draw the timetable.
%    \begin{macrocode}
    \directlua{sp.draw(
        [[\pgfkeysvalueof{/semesterplanner-lua/length}]],
        [[\pgfkeysvalueof{/semesterplanner-lua/width}]])}
}
%    \end{macrocode}
% \end{environment}
%
%    \begin{macrocode}
%</package>
%    \end{macrocode}

% \subsection{semesterplanner-lua.lua}
%    \begin{macrocode}
%<*luaMain>
%    \end{macrocode}
% \begin{macro}{init}
% Initialize global variables to remove previous values (e.g. events from
% the previous timetable)
% \begin{description}
%	\item[|days|] A string with the names of the weekdays for the header
%	\item[|min|] Time where the timetable should start. If empty this is
%	calculated from the events.
%	\item[|max|] Time where the timetable should end. If empty this is
%	calculated from the events.
% \end{description}
%    \begin{macrocode}
function init(days, min, max)
    -- clean up first
    -- global variables
    EVENTS={}
    DAYS = days -- header with names of the days set from tex currently
    DAYSE = {"M","T","W","Th","F"}
    MIN = 25*60 -- bigger than any allowed value could be
    MAX = 0
	MIN_BYPASS = false -- weather min is fixed by the user
    MAX_BYPASS = false -- weather max is fixed by the user

    if(min == "") then
    else
        assert(min:match("^%d+"), "start time has to be an integer representing the HH*60+MM of the desired start time")
        MIN = tonumber(min)
        MIN_BYPASS = true
    end

    if(max == "") then
    else
        assert(max:match("^%d+"), "end time has to be an integer representing the HH*60+MM of the desired end time")
        MAX = tonumber(max)
        MAX_BYPASS = true
    end
end
%    \end{macrocode}
% \end{macro}
% \begin{macro}{addEvent}
% Adds the event to the |EVENTS| array after some validiy checks, modifys MIN/MAX if necessary
%    \begin{macrocode}
-- result are the global variables EVENTS, MIN and MAX
function addEvent(opts)
    if(not checkKeys(opts, {"time", "day", "content", "tikz"})) then
        error("missing argument")
    end

    opts.from,opts.to = dur2Int(opts.time)

    if(not MIN_BYPASS and opts.from < MIN) then MIN = opts.from end
    if(not MAX_BYPASS and opts.to   > MAX) then MAX = opts.to   end

    table.insert(EVENTS, opts)
end
%    \end{macrocode}
% \end{macro}
% \begin{macro}{draw}
%    \begin{macrocode}
-- parameters are all global variables
function draw(length, width)
    print("length", length)
    print("width", width)
    -- copy relevant variables for working on local copies
    local events = copy_array(EVENTS)
    local days = prepareDays(DAYS)
    local daysE = copy_array(DAYSE)
    local min, minH, max, maxH = prepareMinMax(MIN, MAX)

    assert(length:match("%d*%.?%d*"), "Length must be a valid length measured in cm")
    length = tonumber(length)

    textwidth = width

    tex.print([[\begin{tikzpicture}]])
    tex.print([[\tikzset{defStyle/.style={font=\tiny,anchor=north west,fill=blue!50,draw=black,rectangle}}]])

    -- print the tabular with the weekday headers
    tex.print(string.format([[\foreach \week [count=\x from 0, evaluate=\x as \y using \x+0.5] in {%s}{ ]], table.concat(days, ",")))
    tex.print(string.format([[\node[anchor=south] at (\y/%d* %s, 0) {\week};]], #days, textwidth))
    tex.print(string.format([[\draw (\x/%d * %s, 0cm) -- (\x/%d * %s, %dcm);]], #days, textwidth, #days, textwidth, -length))
    tex.print("}")
    tex.print(string.format([[\draw (%s, 0) -- (%s,%dcm);]], textwidth, textwidth, -length))

    for i=minH,maxH do
        tex.print(string.format([[\node[anchor=east] at (0,%fcm ) {%d:00};]], minuteToFrac(i*60,min,max)*-length, i))
        tex.print(string.format([[\draw (0,%fcm ) -- (%s,%fcm );]], minuteToFrac(i*60,min,max)*-length, textwidth, minuteToFrac(i*60,min,max)*-length))
    end

    local d
    local red = 0.3333 -- calculated in em from inner sep
    local red_y = 0.25 -- calculated in em
    for _,e in ipairs(events) do
        d = search_array(daysE, e.day) - 1
        tex.print(string.format(
            [[\node[defStyle,text width=-%fem+%f%s/%d, text depth=%fcm-%fem, text height=%fem, %s] at (%f*%s,%fcm) {%s};]],
            2*red, -- text width
            e.scale_width, -- text width
            textwidth,
            #days, -- text width
            length*(e.to-e.from)/(max-min), -- text depth
            2*red+red_y, -- text depth
            red_y, -- text height
            e.tikz, -- free tikz code
            (d+e.offset)/#days, -- xcoord
            textwidth,
            minuteToFrac(e.from,min,max)*-length, -- ycoord
            e.content -- content
            )
        )
    end
    tex.print([[\end{tikzpicture}]])
end
%    \end{macrocode}
% \end{macro}
% \begin{macro}{searchArray}
%    \begin{macrocode}

function search_array(t, s)
    for k,v in ipairs(t) do
        if(v == s) then return k end
    end
    return nil
end

%    \end{macrocode}
% \end{macro}
% \begin{macro}{minuteToFrac}
%    \begin{macrocode}
function minuteToFrac(minute, min, max)
    return (minute-min)/(max-min)
end
%    \end{macrocode}
% \end{macro}
% \begin{macro}{prepareMinMax}
%    \begin{macrocode}

function prepareMinMax(min, max)
    local minH = math.floor(min/60)
    local maxH = math.ceil(max/60)
    local min = minH*60
    local max = maxH*60
    return min, minH, max, maxH
end
%    \end{macrocode}
% \end{macro}
% \begin{macro}{checkKeys}
%    \begin{macrocode}

function checkKeys(t, k)
    for _,x in ipairs(k) do
        if(t[x] == nil) then
            return false
        end
    end
    return true
end

%    \end{macrocode}
% \end{macro}
% \begin{macro}{dur2Int}
%    \begin{macrocode}
function dur2Int(clk)
    local f1,f2, t1,t2 = clk:match("^(%d%d?):(%d%d)-(%d%d?):(%d%d)$")
    if(f1 ~= nil and f2 ~= nil and t1 ~= nil and t2 ~= nil) then
        f1 = tonumber(f1) f2 = tonumber(f2)
        t1 = tonumber(t1) t2 = tonumber(t2)
        assert(f1 >= 0 and f1 < 24, "Hours have to be >= 0 && < 24")
        assert(f2 >= 0 and f2 < 60, "Mins have to be >= 0 && < 60")
        assert(t1 >= 0 and t1 < 24, "Hours have to be >= 0 && < 24")
        assert(t2 >= 0 and t2 < 60, "Mins have to be >= 0 && < 60")
        return f1*60 + f2, t1*60 + t2
    else
        error("clk string \"" .. clk .. "\" was no valid clock string")
    end
end
%    \end{macrocode}
% \end{macro}
% \begin{macro}{prepareDays}
%    \begin{macrocode}

function prepareDays(days)
    local ret = {}
    for m in days:gmatch("[^,]+") do
        table.insert(ret, m)
    end
    return ret
end
%    \end{macrocode}
% \end{macro}
% \begin{macro}{copyArray}
%    \begin{macrocode}

function copy_array(obj)
    if type(obj) ~= 'table' then return obj end
    local res = {}
    for k, v in pairs(obj) do
        local c = copy_array(v)
        res[copy_array(k)] = c
    end
    return res
end
%    \end{macrocode}
% \end{macro}
% Prepare the module semesterplannerLua for exporting (only the functions
% that should be public)
%    \begin{macrocode}

semesterplannerLua = {
    init = init,
    addEvent = addEvent,
    draw = draw
}
return semesterplannerLua
%</luaMain>
%    \end{macrocode}
%\Finale
