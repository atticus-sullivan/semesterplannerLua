% \iffalse meta-comment
% vim: textwidth=75
% vim: set tabstop=4 shiftwidth=4 expandtab
%<*internal>
\iffalse
%</internal>
%<*readme>
|
-------------------:| -----------------------------------------------------
semesterplanner-lua:| Semesterplanner package in lua with tikz only
             Author:| Lukas Heindl
             GitLab:| https://gitlab.com/AtticusSullivan/semesterplanner-lua
            License:| Released under the LaTeX Project Public License v1.3c or later
                See:| http://www.latex-project.org/lppl.txt


Short description:
Some text about the package: probably the same as the abstract.
%</readme>
%<*internal>
\fi
\def\nameofplainTeX{plain}
\ifx\fmtname\nameofplainTeX\else
  \expandafter\begingroup
\fi
%</internal>
%<*install>
\input docstrip.tex
\keepsilent
\askforoverwritefalse
\preamble
-------------------:| -----------------------------------------------------
semesterplanner-lua:| Semesterplanner package in lua with tikz only
             Author:| Lukas Heindl
             GitLab:| https://gitlab.com/AtticusSullivan/semesterplanner-lua
            License:| Released under the LaTeX Project Public License v1.3c or later
                See:| http://www.latex-project.org/lppl.txt

\endpreamble
\postamble

Copyright (C) 2021 by Lukas Heindl

This work may be distributed and/or modified under the
conditions of the LaTeX Project Public License (LPPL), either
version 1.3c of this license or (at your option) any later
version.  The latest version of this license is in the file:

http://www.latex-project.org/lppl.txt

This work is "maintained" (as per LPPL maintenance status) by
(not set).

This work consists of the file semesterplanner-lua.dtx and a Makefile.
Running "make" generates the derived files README, semesterplanner-lua.pdf and semesterplanner-lua.sty.
Running "make inst" installs the files in the user's TeX tree.
Running "make install" installs the files in the local TeX tree.

\endpostamble

\usedir{tex/latex/semesterplanner-lua}
\generate{
  \file{\jobname.sty}{\from{\jobname.dtx}{package}}
}
\usedir{tex/latex/semesterplanner-lua}
\generate{
  \nopreamble\nopostamble
  \file{\jobname-timetable.lua}{\from{\jobname.dtx}{luaTimetable}}
}
\usedir{tex/latex/semesterplanner-lua}
\generate{
  \nopreamble\nopostamble
  \file{\jobname-calendar.lua}{\from{\jobname.dtx}{luaApp}}
}
%</install>
%<install>\endbatchfile
%<*internal>
\usedir{source/latex/semesterplanner-lua}
\generate{
  \file{\jobname.ins}{\from{\jobname.dtx}{install}}
}
\nopreamble\nopostamble
\usedir{doc/latex/semesterplanner-lua}
\generate{
  \file{README.txt}{\from{\jobname.dtx}{readme}}
}
\ifx\fmtname\nameofplainTeX
  \expandafter\endbatchfile
\else
  \expandafter\endgroup
\fi
%</internal>
% \fi
%
% \iffalse
%<*driver>
\ProvidesFile{semesterplanner-lua.dtx}
%</driver>
%<*driver>
\documentclass{ltxdoc}
\usepackage[a4paper,margin=25mm,left=50mm,nohead]{geometry}
\usepackage[numbered]{hypdoc}
\usepackage{array}
\usepackage{\jobname}
\EnableCrossrefs
\CodelineIndex
\RecordChanges
\begin{document}
  \DocInput{\jobname.dtx}
\end{document}
%</driver>
%<package>\NeedsTeXFormat{LaTeX2e}[1999/12/01]
%<package>\ProvidesPackage{semesterplanner-lua}
%<*package>
    [2021/10/07 v0.2 Semesterplanner package in lua with tikz only]

\RequirePackage{tikz}
\usetikzlibrary{calendar, positioning, shapes.misc, backgrounds}
\RequirePackage{pgfkeys}
\RequirePackage{xcolor}
\RequirePackage{fontawesome}
\usepackage{luapackageloader} % use the default lua path as well
%</package>
% \fi
%
% \GetFileInfo{\jobname.dtx}
% \DoNotIndex{\newcommand,\newenvironment,\\,\begin,\end,\def,\definecolor,\directlua,\endinput,\faBullhorn,\faCamera,\faFlag,\faQuestion,\faTimesCircle,\faWarning,\faWindows,\faYoutubePlay,\node,\pgfkeys,\pgfkeysvalueof,\protected,\raggedright,\textbf,\textcolor,\textwidth,\unexpanded,\"}
%
%\title{\textsf{semesterplanner-lua} --- Semesterplanner package in lua
%with tikz only\thanks{This file describes version \fileversion, last
%revised \filedate.}
%}
%\author{Lukas Heindl\\\faGitlab:
%\url{https://gitlab.com/AtticusSullivan/semesterplanner-lua}}
%\date{Released \filedate}
%
%
%\maketitle
%
%\changes{v1.00}{2021/10/07}{First public release}
%
% \begin{abstract}
% This package provides a mean to easily print a timetable e.g. for a
% semesterplan. The reason for this package to exist is that I wanted to
% reimplement \url{https://github.com/nlschn/semesterplanner/} with
% printing the timetable with |tikz| only (which is more easily to be
% modified) and with the ability to make entries spanning only a fraction
% of the column (for showing simultanious events).
% 
% Documents using this package need to be compiled with
% LuaLaTeX. The package requires |xcolor|, |fontawesome|, |tikz| (and |pgfkeys|).
% \end{abstract}
%
% \tableofcontents
%
% \section{Usage}
% \subsection{timetable}
%
% \DescribeEnv{timetable} |\begin{timetable}[opts]\ldots\end{timetable}|\\
% |opts| are of course optional arguments:
% \begin{description}
%     \item[|days|] List of the names of the days that should be set as
%     column names. Note that if you specify only 4 names only these 4
%     columns will be printed (with the first day being identified as
%     Monday)
%     \textit{Default: |Mon,Thue,Wend,Thur,Fri|}
%     \item[|start time|] Explicit start-time of the timetable given in minutes
%     (|HH*60 + MM|). Can be set as |start time/.evaluated={HH*60 + MM}|.
%     If this is empty, the start time is derived from the given events.
%     \textit{Default: |""|}
%     \item[|end time|] Equivalent to |start-time|
%     \textit{Default: |""|}
%     \item[|width|] Give the width of the timetable. (can be given e.g. as
%     |\textwidth| as this is directly given to tikz).
%     \textit{Default:} |\textwidth|
%     \item[|length|] Give the length of the timetable (measured in |cm|)
%     (has to be a straight number since this is needed in calculation)
%     \textit{Default: 10}
% \end{description}
% 
% This is the core environment of this package. Within it you can use
% |\lecture|, |\seminar|, |\tutorial|, |\officehour| and |\meeting|. All
% these commands are only defined inside the |timetable| environment, and
% have the same structure.
%
% \DescribeMacro{\lecture}    |\lecture   {Name}{Lecturer}{Place}{Day}{Time}{Priority}{Event-code}|
%
% \DescribeMacro{\tutorial}   |\tutorial  {Name}{Lecturer}{Place}{Day}{Time}{Priority}{Event-code}|
%
% \DescribeMacro{\seminar}    |\seminar   {Name}{Lecturer}{Place}{Day}{Time}{Priority}{Event-code}|
%
% \DescribeMacro{\officehour} |\officehour{Name}{Lecturer}{Place}{Day}{Time}{Priority}{Event-code}|
%
% \DescribeMacro{\meeting}    |\meeting   {Name}{Lecturer}{Place}{Day}{Time}{Priority}{Event-code}|
% \vspace*{2em}
%
% \begin{description}
%   \item[|Name|] Give the name of the lecture
%   \item[|Lecturer|] Give the name of the lecturer
%   \item[|Place|] Give the place of the event (most probably the room or
%   an online plattform, see \ref{icons})
%   \item[|Day|] The weekday on which the event takes place. Has to be one of
%   |M|, |T|, |W|, |Th|, |F| for \textbf{M}onday, \textbf{T}huesday,
%   \textbf{W}ednesday, \textbf{Th}ursday, \textbf{F}riday.  Might become
%   customizable in a future version.  \item[|Time|] The timespan of the
%   event formatted as |HH:MM-HH:MM| (24H clock)
%   \item[|Priority|] The priority of the event (see \ref{icons})
%   \item[|Event-code|] Free customizable event code. See the documentation
%   at the end for keys that can be used here (all keys in |/event|). To
%   simply pass arguments to the tikz-node that is being created for the
%   event use |tikz/.append={your arguments}| (be careful with |text width|,
%   |text height|, |text depth| as these keys are being used for
%   the dimensions of the node as well as with |anchor|)
% \end{description}
% The entries |Day| and |Time| are mandatory since they are needed for the
% positioning of the node. All others are merely necessary for the content
% of the node and are therefore nor mandatory.
%
% \subsubsection{Special Notes}
% Note that the |length| argument does specify the length of the timetable
% without taking account of the column headers.
%
% Same goes for the |width| parameter regarding the labels containing the
% time on the right. Since in this case any tex-lenght is allowed, you can
% simply try to subtract the length of the clock label using something like
% |\settowidth{\length}{12:30}| to set a length to the length of a clock
% label and then subtract this from the length you want to specify.
%
% \hangindent=4.5em \hangafter=1 \textbf{Hint:} The content of the
% environment isn't processed by this package. Only the event commands (so
% to speak |\lecture|,|\tutorial|,|\seminar|,|\officehour|,|\meeting| are
% relevant.  All other contents are set immediately before the timetable.
% Therefore, if you wan to add e.g. a |\hspace*{10cm}| to shift the
% timetable to the left, the last line of the env would be the place to do
% so (there musn't be an empty line below since otherwise a new paragraph
% is started).
%
% \newpage
% \subsubsection{Example}
% |\begin{timetable}[|\\
% |        days={Mon,Thue,Wend,Thur,Fri}, start|\\
% |        time/.evaluated={11*60}, end time/.evaluated={15*60}|\\
% |    ]|\\
% |    \lecture[title={TestingLectureLongOne},speaker={Heindl},location={RN1},day={W},time={12:30-13:30}]|\\
% |    \lecture[title={TestingLectureLongOne},speaker={Heindl},location={RN1},day={Th},time={12:30-13:30},offset=0.5,scale width=0.5]|\\
% |    \lecture[title={TestingLectureLongOne},speaker={Heindl},location={\zoom},day={T},time={12:30-13:30},prio={\phigh}]|\\
% |\end{timetable}|\\
%
% \begin{timetable}[days={Mon,Thue,Wend,Thur,Fri}, start
% time/.evaluated={11*60}, end time/.evaluated={15*60}]
%   \lecture[title={TestingLectureLongOne},speaker={Heindl},location={RN1},day={W},time={12:30-13:30}]
%   \lecture[title={TestingLectureLongOne},speaker={Heindl},location={RN1},day={Th},time={12:30-13:30},offset=0.5,scale width=0.5]
%   \lecture[title={TestingLectureLongOne},speaker={Heindl},location={\zoom},day={T},time={12:30-13:30},prio={\phigh}]
%   \hspace*{-.2\textwidth}
% \end{timetable}
%
% \subsection{Icons}\label{icons}
% This package defines some modified fontawesome icons (they are being
% encircled with a white circle for better readability).\\
% \begin{tabular}{rl|rl}
%     |\zoom| & \textcolor{DodgerBlue}{\faCamera} &
%     |\teams| & \textcolor{DodgerBlue}{\faWindows}
%     \\
%     |\BBB| & \textcolor{DodgerBlue}{\faBold} &
%     |\youtube| & \textcolor{DodgerBlue}{\faYoutubePlay}
%     \\\hline
%     |\pmandatory| & \textcolor{red}{\faWarning} &
%     |\phigh| & \textcolor{red}{\faFlag}
%     \\
%     |\pmid| & \textcolor{yellow}{\faFlag} &
%     |\plow| & \textcolor{green}{\faFlag}
%     \\
%     |\pnone| & \textcolor{gray}{\faTimesCircle}
%     \\ \hline
%     |\tbd| & \faQuestion &
%     |\tba| & \faBullhorn
% \end{tabular}
%
%
%
%\StopEventually{
%  \PrintChanges
%  \PrintIndex
%}
%
% \newpage
% \section{Implementation}
% This package uses |semesterplanner-lua| as prefix/directory where
% possible. Since this is not possible for latex macro names, in this
% occasions |semesterplannerLua@| is used as prefix.
% \subsection{semesterplanner-lua.sty}
% \subsubsection{Global Stuff}
%    \begin{macrocode}
%<*package>
%    \end{macrocode}
% Define some colors for the course types (can be globally overwritten)
%    \begin{macrocode}
\definecolor{seminar}{rgb}{1.0, 0.8, 0.0}
\definecolor{lecture}{rgb}{0.2, 0.7, 1.0}
\definecolor{tutorial}{rgb}{0.0, 0.8, 0.0}
\definecolor{meeting}{rgb}{0.8, 0.0, 0.0}
\definecolor{officehour}{rgb}{0.0, 0.4, 0.6}
\definecolor{DodgerBlue}{HTML}{1E90FF}
%    \end{macrocode}
% \begin{macro}{\semesterplannerLua@encircle}
% This macro puts a circle arround its argument for better readability. In
% this package this is used for the fontawesome symbols.
%    \begin{macrocode}
    \newcommand*{\semesterplannerLua@encircle}[1]{
        \begin{minipage}[b][1em][c]{1.5em}
            \begin{tikzpicture}
                \node[fill,circle,inner sep=1pt, color = white] {#1};
            \end{tikzpicture}   
        \end{minipage}
    }
%    \end{macrocode}
% \end{macro}
%    Commands for exams
%   \begin{macro}{\oral}
%    \begin{macrocode}
\protected\def\oral{\faComment}
%    \end{macrocode}
%   \end{macro}
%   \begin{macro}{\written}
%    \begin{macrocode}
\protected\def\written{\faPencil}
%    \end{macrocode}
%   \end{macro}
%   Commands for symbols of priority
%   \begin{macro}{\pmandatory}
%    \begin{macrocode}
    \protected\def\pmandatory{\semesterplannerLua@encircle{\textcolor{red}{\faWarning}}}
%    \end{macrocode}
%   \end{macro}
%   \begin{macro}{\phigh}
%    \begin{macrocode}
    \protected\def\phigh{\semesterplannerLua@encircle{\textcolor{red}{\faFlag}}}
%    \end{macrocode}
%   \end{macro}
%   \begin{macro}{\pmid}
%    \begin{macrocode}
    \protected\def\pmid{\semesterplannerLua@encircle{\textcolor{yellow}{\faFlag}}}
%    \end{macrocode}
%   \end{macro}
%   \begin{macro}{\plow}
%    \begin{macrocode}
    \protected\def\plow{\semesterplannerLua@encircle{\textcolor{green}{\faFlag}}}
%    \end{macrocode}
%   \end{macro}
%   \begin{macro}{\pnone}
%    \begin{macrocode}
    \protected\def\pnone{\semesterplannerLua@encircle{\textcolor{gray}{\faTimesCircle}}}
%    \end{macrocode}
%   \end{macro}

%   Commands for online platforms.
%   \begin{macro}{\teams}
%    \begin{macrocode}
    \protected\def\teams{\semesterplannerLua@encircle{\textcolor{DodgerBlue}{\faWindows}}}
%    \end{macrocode}
%   \end{macro}
%   \begin{macro}{\zoom}
%    \begin{macrocode}
    \protected\def\zoom{\semesterplannerLua@encircle{\textcolor{DodgerBlue}{\faCamera}}}
%    \end{macrocode}
%   \end{macro}
%   \begin{macro}{\youtube}
%    \begin{macrocode}
    \protected\def\youtube{\semesterplannerLua@encircle{\textcolor{red}{\faYoutubePlay}}}
%    \end{macrocode}
%   \end{macro}
%   \begin{macro}{\BBB}
%    \begin{macrocode}
    \protected\def\BBB{\semesterplannerLua@encircle{\textcolor{DodgerBlue}{\faBold}}}
%    \end{macrocode}
%   \end{macro}

%   Command for "To be determined" and "To be Announced"
%   \begin{macro}{\tbd}
%    \begin{macrocode}
    \protected\def\tbd{\faQuestion}
%    \end{macrocode}
%   \end{macro}
%   \begin{macro}{\tba}
%    \begin{macrocode}
    \protected\def\tba{\faBullhorn}
%    \end{macrocode}
%   \end{macro}


% Load the lua modules
%    \begin{macrocode}
\directlua{sp = require("semesterplanner-lua-timetable.lua")}
\directlua{cal = require("semesterplanner-lua-calendar.lua")}
%    \end{macrocode}
% Set all the pgfkeys required for the arguments. To achieve that the
% defaults are restored every time the environment is used, this is
% inside the environment definition. This of course disables all
% possibilities of setting a global default but enables setting local
% defaults for the events
%    \begin{macrocode}
    \pgfkeys{
%    \end{macrocode}
% |/semesterplanner-lua| will be the pgf-path used for this package. Here
% all used keys are set (and initialized with defaults. |timetable/env/|:
% \begin{description}
%   \item[|days|] is a list of
%       strings representing the header names for the day columns in the
%       timetable (adding Sat and Sun (additional entries) will result in
%       two more columns.
%   \item[|start time|] can be used to set a fixed time where the timetable
%       starts (otherwise this is calculated from the entries) to enable
%       this behaviour this key has to be set to |HH*60 + MM| (easy way
%       is by using |start time/.evaluated={HH*60+MM}|)
%   \item[|end time|] equivalent to |start time|
%   \item[|width|] is
%       the horizontal width of the timetable (not including the column
%       headers on the top) this can be a latex length string or
%       |\textwidth| as well.
%   \item[|length|] is the vertical length of the
%       timetable (not including the clock labels on the side)
%       measured in cm (in future versions this may become measured in
%       pts for better interaction with the LaTeX lengths.
% \end{description}
%    \begin{macrocode}
        /semesterplanner-lua/timetable/env/.cd,
        days/.initial={Mon,Thue,Wend,Thur,Fri}, days/.default={Mon,Thue,Wend,Thur,Fri},
        %
        start time/.initial=, start time/.default=,
        end time/.initial=, end time/.default=,
        %
        width/.initial=\textwidth, width/.default=\textwidth,
        length/.initial=10, length/.default=10,
        %
%    \end{macrocode}
% |timetable/event/|:
% \begin{description}
%   \item[|content|] is the content of the event (is passed on without any
%           formatting). Since this is passed to lua without modification
%           its value must be an unexpanded string (lua will simply print
%           it so the eventually the string will be evaluated)
%   \item[|time|] is a |HH:MM-HH:MM| string representing start- and
%       end-time of the event. Used in constructing the content as well
%   \item[|day|] is either |M|,|T|,|W|,|Th| or |F| specifying the day on
%       which the event takes place
%   \item[|tikz|] this key allows the user to manually pass options to the
%       node created for this event
%   \item[|scale width|] allows to scale the width of the event to be able
%       to draw overlapping events besides each other. Will usually be a
%       value between |0| and |1|.
%   \item[|offset|] same goal like |scale width| but shifts the event node
%       by the given value to the right. (Given as value between |0| and
%       |1| indicating how many columns the event should be shifted)
%   \item[|textcolor|] foreground color of the content text
%   \item[|title|] title (set in bold by default)
%   \item[|speaker|] 
%   \item[|location|] 
%   \item[|prio|] 
%   \item[|formatter|] this is special
% \end{description}
%    \begin{macrocode}
        /semesterplanner-lua/timetable/event/.cd,
        % event arguments
        content/.initial=, content/.default=,
        %
        time/.initial=, time/.default=,
        day/.initial=, day/.default=,
        %
        tikz/.initial=, tikz/.default=,
        scale width/.initial=1, scale width/.default=1,
        offset/.initial=0, offset/.default=0,
        %
        textcolor/.initial=, textcolor/.default=,
        title/.initial=, title/.default=,
        speaker/.initial=, speaker/.default=,
        location/.initial=, location/.default=,
        prio/.initial=, prio/.default=,
        formatter/.initial=timetableformatter, formatter/.default=timetableformatter,
        %
%    \end{macrocode}
% |calendar/|:
% \begin{description}
%   \item[|draw|] 
%   \item[|room|] 
%   \item[|prio|] 
%   \item[|course|] 
%   \item[|desc|] 
%   \item[|start|] 
%   \item[|end|] 
%   \item[|tikz|] 
%   \item[|period|] 
% \end{description}
%    \begin{macrocode}
        /semesterplanner-lua/calendar/.cd,
        draw/.initial={true}, draw/.default={true},
        room/.initial={}, room/.default={},
        time/.initial={}, time/.default={},
        prio/.initial={}, prio/.default={},
        course/.initial={}, course/.default={},
        desc/.initial={}, desc/.default={},
        type/.initial={}, type/.default={},
        date/.initial={}, date/.default={},
        end/.initial={}, end/.default={},
        tikz/.initial={}, tikz/.default={},
        period/.initial={nil}, period/.default={nil},
    }
%    \end{macrocode}
%
% \subsubsection{Local Stuff (timetable-env local)}
% \begin{environment}{timetable}
% This is the environment doing all the stuff. To gate the positions where
% the corresponding macros can be used (and in terms of pgfkeys for reasons
% of default values) all the macros used are put into the environment.
%    \begin{macrocode}
\newenvironment{timetable}[1][]{
    \section*{\faClockO~Timetable}
%    \end{macrocode}
%
%   Read the argumens given by the user after restoring the defaults
%   (Restoring currently makes no sense, since they are created a few
%   lines above anyways, but creation might be moved outside the
%   environment some day.\\
%   Afterwards the lua module is beeing initialized (erase data from
%   possible previous runs.
%    \begin{macrocode}
    \pgfkeys{/semesterplanner-lua/timetable/env/.cd, days,start time,end time, width,length, #1}
    \directlua{sp.init(
        "\pgfkeysvalueof{/semesterplanner-lua/timetable/env/days}",
        "\pgfkeysvalueof{/semesterplanner-lua/timetable/env/start time}",
        "\pgfkeysvalueof{/semesterplanner-lua/timetable/env/end time}")}
%    \end{macrocode}
%   \begin{macro}{\semesterplanner@event}
% Is used to pass the event to the lua engine which in turn will collect
% the event to draw it in the end. For that the arguments given are parsed
% after restoring the pgf keys to their default values. The optional
% argument herby is a sequence of pgf keys, the second argument is
% a string representing the content (this MUST be unexpanded since
% this is passed to lua which in turn will pass it unmodified back)
%    \begin{macrocode}
    \newcommand{\semesterplannerLua@event}[1][]{
        \pgfkeys{/semesterplanner-lua/timetable/event/.cd,content,time,day,tikz,scale
		width,offset,textcolor,title,speaker,location,prio,formatter, ##1}
        \directlua{
            sp.addEvent{
                time="\pgfkeysvalueof{/semesterplanner-lua/timetable/event/time}",
                day="\pgfkeysvalueof{/semesterplanner-lua/timetable/event/day}",
                tikz=[[\pgfkeysvalueof{/semesterplanner-lua/timetable/event/tikz}]],
                offset=\pgfkeysvalueof{/semesterplanner-lua/timetable/event/offset},
                scale_width=\pgfkeysvalueof{/semesterplanner-lua/timetable/event/scale width},
                formatter=\pgfkeysvalueof{/semesterplanner-lua/timetable/event/formatter},
                textcolor=[[\pgfkeysvalueof{/semesterplanner-lua/timetable/event/textcolor}]],
                title=[[\pgfkeysvalueof{/semesterplanner-lua/timetable/event/title}]],
                speaker=[[\pgfkeysvalueof{/semesterplanner-lua/timetable/event/speaker}]],
                location=[[\pgfkeysvalueof{/semesterplanner-lua/timetable/event/location}]],
                prio=[[\pgfkeysvalueof{/semesterplanner-lua/timetable/event/prio}]],
            }
        }
    }
%    \end{macrocode}
%   \end{macro}
% Short-hand macros for different events using the corresponding
% background color
%
%   \begin{macro}{\lecture}
%    \begin{macrocode}
    \newcommand{\lecture}[1][]{
        \semesterplannerLua@event[tikz={fill=lecture,}, textcolor=white, ##1]
    }
%    \end{macrocode}
%   \end{macro}
%   \begin{macro}{\seminar}
%    \begin{macrocode}
    \newcommand{\seminar}[1][]{
        \semesterplannerLua@event[tikz={fill=seminar,}, textcolor=white, ##1]
    }
%    \end{macrocode}
%   \end{macro}
%   \begin{macro}{\tutorial}
%    \begin{macrocode}
    \newcommand{\tutorial}[1][]{
        \semesterplannerLua@event[tikz={fill=tutorial,}, textcolor=white, ##1]
    }
%    \end{macrocode}
%   \end{macro}
%   \begin{macro}{\meeting}
%    \begin{macrocode}
    \newcommand{\meeting}[1][]{
        \semesterplannerLua@event[tikz={fill=meeting,}, textcolor=white, ##1]
    }
%    \end{macrocode}
%   \end{macro}
%   \begin{macro}{\officehour}
%    \begin{macrocode}
    \newcommand{\officehour}[1][]{
        \semesterplannerLua@event[tikz={fill=officehour,}, textcolor=white, ##1]
    }
%    \end{macrocode}
%   \end{macro}
%    \begin{macrocode}
}{
%    \end{macrocode}
% At the end of the environment after all events have been collected,
% generate and output the tikz code needed to draw the timetable.
%    \begin{macrocode}
    \directlua{sp.draw(
        [[\pgfkeysvalueof{/semesterplanner-lua/timetable/env/length}]],
        [[\pgfkeysvalueof{/semesterplanner-lua/timetable/env/width}]])}
}
%    \end{macrocode}
% \end{environment}
%
%    \begin{macrocode}

%    \end{macrocode}
% \begin{macro}{printSpCalendar}
% Print a calendar from startDate to endDate (encoded as YYYY-MM-DD) as
% one calendar per month in a matrix with the given amount of columns
%    \begin{macrocode}
\newcommand{\printSpCalendar}[3][3]{\directlua{cal.drawCalendar("#2", "#3", #1)}}
%    \end{macrocode}
% \end{macro}
%    \begin{macrocode}

\newenvironment{appointments}[2][Room]{
    \directlua{cal.init(#2)}
    \newcommand{\appointment}[1][]{
        \pgfkeys{/semesterplanner-lua/calendar/.cd,draw,room,time,prio,course,desc,date,end,tikz,period,type, ##1}
        \directlua{
            cal.addAppointment
            {
                draw=\pgfkeysvalueof{/semesterplanner-lua/calendar/draw},
                room=[[\pgfkeysvalueof{/semesterplanner-lua/calendar/room}]],
                time=[[\pgfkeysvalueof{/semesterplanner-lua/calendar/time}]],
                prio=[[\pgfkeysvalueof{/semesterplanner-lua/calendar/prio}]],
                course=[[\pgfkeysvalueof{/semesterplanner-lua/calendar/course}]],
                desc=[[\pgfkeysvalueof{/semesterplanner-lua/calendar/desc}]],
                date=[[\pgfkeysvalueof{/semesterplanner-lua/calendar/date}]],
                endDate=[[\pgfkeysvalueof{/semesterplanner-lua/calendar/end}]],
                tikz=[[\pgfkeysvalueof{/semesterplanner-lua/calendar/tikz}]],
                period=\pgfkeysvalueof{/semesterplanner-lua/calendar/period}
            }
        }
    }
    \section*{\faCalendar~Appointments}
    \begin{tabular}{rlllll}
        \textbf{Date}&\textbf{Time}&\textbf{Course}&\textbf{Description}&\textbf{#1}&\textbf{Prio.}\\
}{
    \end{tabular}
}

\newenvironment{exams}[1]{
    \directlua{cal.init(#1)}
    \newcommand{\exam}[1][]{
        \pgfkeys{/semesterplanner-lua/calendar/.cd,draw,room,time,prio,course,desc,date,end,tikz,period,type, ##1}
        \directlua{
            cal.addExam
            {
                draw=\pgfkeysvalueof{/semesterplanner-lua/calendar/draw},
                room=[[\pgfkeysvalueof{/semesterplanner-lua/calendar/room}]],
                time=[[\pgfkeysvalueof{/semesterplanner-lua/calendar/time}]],
                course=[[\pgfkeysvalueof{/semesterplanner-lua/calendar/course}]],
                desc=[[\pgfkeysvalueof{/semesterplanner-lua/calendar/desc}]],
                date=[[\pgfkeysvalueof{/semesterplanner-lua/calendar/date}]],
                tikz=[[\pgfkeysvalueof{/semesterplanner-lua/calendar/tikz}]],
                type=[[\pgfkeysvalueof{/semesterplanner-lua/calendar/type}]],
            }
        }
    }
    \section*{\faStickyNoteO~Exams}
    \begin{tabular}{rllll}
        \textbf{Date}&\textbf{Time}&\textbf{Course}&\textbf{Type}&\textbf{Note}\\
}{
    \end{tabular}
}

\newenvironment{deadlines}[1]{
    \directlua{cal.init(#1)}
    \newcommand{\deadline}[1][]{
        \pgfkeys{/semesterplanner-lua/calendar/.cd,draw,room,time,prio,course,desc,date,end,tikz,period,type, ##1}
        \directlua{
            cal.addDeadline
            {
                draw=\pgfkeysvalueof{/semesterplanner-lua/calendar/draw},
                course=[[\pgfkeysvalueof{/semesterplanner-lua/calendar/course}]],
                desc=[[\pgfkeysvalueof{/semesterplanner-lua/calendar/desc}]],
                date=[[\pgfkeysvalueof{/semesterplanner-lua/calendar/date}]],
                tikz=[[\pgfkeysvalueof{/semesterplanner-lua/calendar/tikz}]],
                prio=[[\pgfkeysvalueof{/semesterplanner-lua/calendar/prio}]],
            }
        }
    }
    \section*{\faStickyNoteO~Deadlines}
    \begin{tabular}{rlll}
        \textbf{Date}&\textbf{Course}&\textbf{Description}&\textbf{Prio}\\
}{
    \end{tabular}
}
%    \end{macrocode}
%    \begin{macrocode}
%</package>
%    \end{macrocode}

% \subsection{semesterplanner-lua-timetable.lua}
%    \begin{macrocode}
%<*luaTimetable>
%    \end{macrocode}
% \begin{macro}{init}
% Initialize global variables to remove previous values (e.g. events from
% the previous timetable)
% \begin{description}
%   \item[|days|] A string with the names of the weekdays for the header
%   \item[|min|] Time where the timetable should start. If empty this is
%   calculated from the events.
%   \item[|max|] Time where the timetable should end. If empty this is
%   calculated from the events.
% \end{description}
%    \begin{macrocode}
function init(days, min, max)
    -- clean up first
    -- global variables
    EVENTS={}
    DAYS = days -- header with names of the days set from tex currently
    DAYSE = {"M","T","W","Th","F"}
    MIN = 25*60 -- bigger than any allowed value could be
    MAX = 0
    MIN_BYPASS = false -- weather min is fixed by the user
    MAX_BYPASS = false -- weather max is fixed by the user

    if(min == "") then
    else
        assert(min:match("^%d+"), "start time has to be an integer representing the HH*60+MM of the desired start time")
        MIN = tonumber(min)
        MIN_BYPASS = true
    end

    if(max == "") then
    else
        assert(max:match("^%d+"), "end time has to be an integer representing the HH*60+MM of the desired end time")
        MAX = tonumber(max)
        MAX_BYPASS = true
    end
end

function defaultFormatter(opts)
    ret = ""
    for k,v in pairs(opts) do
        if type(k) == "string" then k = k:gsub("[_^]", "") end
        if type(v) == "string" then v = v:gsub("[_^]", "") end
        ret = string.format("%s, %s: %s", ret, tostring(k), tostring(v))
    end
    print(ret)
    return ret
end

function timetableformatter(opts)
    return string.format(
        [[\textcolor{%s}{\textbf{%s}\\[.2em]\raggedright{%s}\\[0.5em]\raggedright{%s}\raggedright{%s}\\[0.5em]\raggedright{%s}}]],
            opts.textcolor, opts.title, opts.speaker, opts.prio, opts.location, opts.time)
end
%    \end{macrocode}
% \end{macro}
% \begin{macro}{addEvent}
% Adds the event to the |EVENTS| array after some validiy checks, modifys MIN/MAX if necessary
%    \begin{macrocode}
-- result are the global variables EVENTS, MIN and MAX
function addEvent(opts)
    print("Reading event on line ", tex.inputlineno)
    opts.inputlineno = tex.inputlineno
    if(not checkKeys(opts, {"time", "day", "tikz"})) then
        error("missing argument")
    end

    if opts.content == nil then
        if opts.formatter == nil then
            opts.content = defaultFormatter(opts)
        else
            opts.content = opts.formatter(opts)
        end
    end

    opts.from,opts.to = dur2Int(opts.time)

    if(not MIN_BYPASS and opts.from < MIN) then MIN = opts.from end
    if(not MAX_BYPASS and opts.to   > MAX) then MAX = opts.to   end
    assert(opts.from < opts.to, "From has to be before to")

    table.insert(EVENTS, opts)
end
%    \end{macrocode}
% \end{macro}
% \begin{macro}{draw}
% Draws the tikz-timetable with the global variables |EVENTS|, |MIN|,
% |MAX|, |DAYSE| and |DAYS|. In addition |length| and |width| are given as
% direct parameters.
%    \begin{macrocode}
-- parameters are all global variables
function draw(length, width)
    -- copy relevant variables for working on local copies
    local events = copy_array(EVENTS)
    local days = prepareDays(DAYS)
    local daysE = copy_array(DAYSE)
    local min, minH, max, maxH = prepareMinMax(MIN, MAX)

    assert(length:match("%d*%.?%d*"), "Length must be a valid length measured in cm")
    length = tonumber(length)

    textwidth = width

    tex.print([[\begin{tikzpicture}]])
    tex.print([[\tikzset{defStyle/.style={font=\tiny,anchor=north west,fill=blue!50,draw=black,rectangle}}]])
%    \end{macrocode}
% Draw the grid of the timetable along with clock and day labels
%    \begin{macrocode}
    -- print the tabular with the weekday headers
    tex.print(string.format(
        [[\foreach \week [count=\x from 0, evaluate=\x as \y using \x+0.5] in {%s}{ ]],
        table.concat(days, ",")
        )
    )
    tex.print(string.format(
        [[\node[anchor=south] at (\y/%d* %s, 0) {\week};]], #days, textwidth))
    tex.print(string.format(
        [[\draw (\x/%d * %s, 0cm) -- (\x/%d * %s, %dcm);]],
        #days,
        textwidth,
        #days,
        textwidth, -length
        )
    )
    tex.print("}")
    tex.print(string.format(
        [[\draw (%s, 0) -- (%s,%dcm);]],
        textwidth,
        textwidth,
        -length
        )
    )

    for i=minH,maxH do
        tex.print(string.format(
            [[\node[anchor=east] at (0,%fcm ) {%d:00};]],
            minuteToFrac(i*60,min,max)*-length, i
            )
        )
        tex.print(string.format(
            [[\draw (0,%fcm ) -- (%s,%fcm );]],
            minuteToFrac(i*60,min,max)*-length,
            textwidth,
            minuteToFrac(i*60,min,max)*-length
            )
        )
    end

%    \end{macrocode}
% Draw the nodes of the events
%    \begin{macrocode}
    local d
    local red = 0.3333 -- calculated in em from inner sep
    local red_y = 0.25 -- calculated in em
    for _,e in ipairs(events) do
        if e.from < max and e.to > min then -- only draw if event is in scope (part of the comp is done in addEvent from < to
            if e.to   > max then e.to   = max end
            if e.from < min then e.from = min end
            print("Drawing event on line ", e.inputlineno)
            d = search_array(daysE, e.day) - 1
            tex.print(string.format(
                [[\node[defStyle,text width=-%fem+%f%s/%d, text depth=%fcm-%fem, text height=%fem, %s] at (%f*%s,%fcm) {%s};]],
                2*red, -- text width
                e.scale_width, -- text width
                textwidth,
                #days, -- text width
                length*(e.to-e.from)/(max-min), -- text depth
                2*red+red_y, -- text depth
                red_y, -- text height
                e.tikz, -- free tikz code
                (d+e.offset)/#days, -- xcoord
                textwidth,
                minuteToFrac(e.from,min,max)*-length, -- ycoord
                e.content -- content
                )
            )
        end
    end
    tex.print([[\end{tikzpicture}]])
end
%    \end{macrocode}
% \end{macro}
% \begin{macro}{searchArray}
% Searches an array for a given value and returns the index if found. On
% error |nil| is returned
%    \begin{macrocode}
function search_array(t, s)
    for k,v in ipairs(t) do
        if(v == s) then return k end
    end
    return nil
end

%    \end{macrocode}
% \end{macro}
% \begin{macro}{minuteToFrac}
% Calculates at which fraction of the total duration of |max-min| the time
% |minute| is located
%    \begin{macrocode}
function minuteToFrac(minute, min, max)
    return (minute-min)/(max-min)
end
%    \end{macrocode}
% \end{macro}
% \begin{macro}{prepareMinMax}
% Calculates the next hour of |MIN| (next before) and |MAX| (next after)
% and returns it (the hour) and the corresponding |min|/|max| (same in minutes)
%    \begin{macrocode}
function prepareMinMax(min, max)
    local minH = math.floor(min/60)
    local maxH = math.ceil(max/60)
    local min = minH*60
    local max = maxH*60
    return min, minH, max, maxH
end
%    \end{macrocode}
% \end{macro}
% \begin{macro}{checkKeys}
% Checks if all |k|s are present in table |t|
%    \begin{macrocode}
function checkKeys(t, k)
    for _,x in ipairs(k) do
        if(t[x] == nil) then
            return false
        end
    end
    return true
end
%    \end{macrocode}
% \end{macro}
% \begin{macro}{dur2Int}
% Takes a clock duration formatted as |HH:MM-HH:MM|, splits it, checks for
% validity and returns begin/end time in minutes
%    \begin{macrocode}
function dur2Int(clk)
    local f1,f2, t1,t2 = clk:match("^(%d%d?):(%d%d)-(%d%d?):(%d%d)$")
    if(f1 ~= nil and f2 ~= nil and t1 ~= nil and t2 ~= nil) then
        f1 = tonumber(f1) f2 = tonumber(f2)
        t1 = tonumber(t1) t2 = tonumber(t2)
        assert(f1 >= 0 and f1 < 24, "Hours have to be >= 0 && < 24")
        assert(f2 >= 0 and f2 < 60, "Mins have to be >= 0 && < 60")
        assert(t1 >= 0 and t1 < 24, "Hours have to be >= 0 && < 24")
        assert(t2 >= 0 and t2 < 60, "Mins have to be >= 0 && < 60")
        return f1*60 + f2, t1*60 + t2
    else
        error("clk string \"" .. clk .. "\" was no valid clock string")
    end
end
%    \end{macrocode}
% \end{macro}
% \begin{macro}{prepareDays}
% Splits the comma-sep string |days| into an array
%    \begin{macrocode}
function prepareDays(days)
    local ret = {}
    for m in days:gmatch("[^,]+") do
        table.insert(ret, m)
    end
    return ret
end
%    \end{macrocode}
% \end{macro}
% \begin{macro}{copyArray}
% Returns a copy of the table |obj|
%    \begin{macrocode}

function copy_array(obj)
    if type(obj) ~= 'table' then return obj end
    local res = {}
    for k, v in pairs(obj) do
        local c = copy_array(v)
        res[copy_array(k)] = c
    end
    return res
end
%    \end{macrocode}
% \end{macro}
% Prepare the module semesterplannerLua for exporting (only the functions
% that should be public)
%    \begin{macrocode}

semesterplannerLua = {
    init = init,
    addEvent = addEvent,
    draw = draw
}
return semesterplannerLua
%</luaTimetable>
%    \end{macrocode}

% \subsection{semesterplanner-lua-calendar.lua}
% TODO how to set the paths right in this case
% Include the date module for time date calculations
%    \begin{macrocode}
%<*luaApp>
package.path='/usr/share/lua/5.3/?.lua;/usr/share/lua/5.3/?/init.lua;/usr/lib/lua/5.3/?.lua;/usr/lib/lua/5.3/?/init.lua;./?.lua;./?/init.lua;/home/lukas/.luarocks/share/lua/5.3/?.lua;/home/lukas/.luarocks/share/lua/5.3/?/init.lua'
package.cpath='/usr/lib/lua/5.3/?.so;/usr/lib/lua/5.3/loadall.so;./?.so;/home/lukas/.luarocks/lib/lua/5.3/?.so'

local dateLib = require "date"
%    \end{macrocode}
% \begin{macro}{init}
% Initialize the EVENTS table as some sort of a reset, takes an argument
% wethet the reset should be executed (to enable concatenation)
%    \begin{macrocode}
function init(clear)
    -- clean up first
    -- global variable
    if clear then
        EVENTS = {}
    end
end

function genDot(opts)
    dot = ""
    if opts.draw then
        dot = string.format([[\tikz[baseline=(X.base)]\node (X) [fill opacity=.5,fill=red,circle,inner sep=0mm, %s] {\phantom{D}};]], opts.tikz)
    end
    return dot
end

%    \end{macrocode}
% \end{macro}
% \begin{macro}{addEvent}
% Adds an event to the list, stores the date and how the event
% should be highlighted (tikz code for a node)
%    \begin{macrocode}
function addEvent(opts)
    if opts.draw then
        assert(opts.date ~= nil and opts.tikz ~= nil, "date and tikz has to be given")
        if opts.endDate == nil or opts.endDate == '' then
            table.insert(EVENTS, {date=dateLib(opts.date), tikz=opts.tikz, period=opts.period, endDate=nil})
        else
            table.insert(EVENTS, {date=dateLib(opts.date), tikz=opts.tikz, period=opts.period, endDate=dateLib(opts.endDate)})
        end
    end
end

function addAppointment(opts)
    addEvent(opts)
    dot = genDot(opts)
    tex.print(string.format([[ \textit{%s} & %s & %s%s & %s & %s & %s\\]], opts.date, opts.time, dot, opts.course, opts.desc, opts.room, opts.prio))
end

function addExam(opts)
    addEvent(opts)
    dot = genDot(opts)
    tex.print(string.format([[ \textit{%s} & %s & %s%s & %s & %s \\]], opts.date, opts.time, dot, opts.course, opts.type, opts.desc))
end

function addDeadline(opts)
    addEvent(opts)
    dot = genDot(opts)
    tex.print(string.format([[ \textit{%s} & %s%s & %s & %s \\]], opts.date, dot, opts.course, opts.desc, opts.prio))
end
%    \end{macrocode}
% \end{macro}
% \begin{macro}{drawCalendar}
% Draw the calendar month by month in a matrix with given columns. The
% calendar starts and ends at the given dates (in YYYY-MM-DD or any other
% format the datelib understands)
%    \begin{macrocode}

function drawCalendar(minDate, maxDate, cols)
    minDate = dateLib(minDate)
    maxDate = dateLib(maxDate)
    tex.print([[\begin{tikzpicture}[every calendar/.style={inner sep=2pt, week list, month label above centered, month text={\textcolor{red}{\%mt} \%y-}}] ]])
    tex.print([[\matrix[column sep=1em, row sep=1em]{]])
        local i = 1
        running = true
        while running do
            -- derive end from start, then check if maxDate is reached
            endDate = minDate:copy():addmonths(1):setday(1):adddays(-1)
            if endDate >= maxDate then
                endDate = maxDate
                running = false
            end
            tex.print(string.format(
            [[\calendar (%04d-%02d) [dates=%04d-%02d-%02d to %04d-%02d-%02d] if (Sunday) [red] if (Saturday) [red!50!white] if (equals=\year-\month-\day) [nodes={rectangle,draw}] if (at least=\year-\month-\day) {} else [nodes={strike out, draw}]; ]],
                    minDate:getyear(), minDate:getmonth(), minDate:getyear(), minDate:getmonth(), minDate:getday(), endDate:getyear(), endDate:getmonth(), endDate:getday()))

            minDate:addmonths(1)
            minDate:setday(1)

            if i % cols == 0 or not running then
                tex.print([[\\]])
            else
                tex.print([[&]])
            end
            i = i + 1
        end
        tex.print([[ }; ]])

%    \end{macrocode}
% Draw highlighting on a background layer so that the calendar
% is not overdrawn
%    \begin{macrocode}
        local usedDates = {}
        tex.print([[\begin{scope}[on background layer] ]])
        for i,ele in ipairs(EVENTS) do
            while ele.date <= maxDate and (ele.endDate == nil or ele.date <= ele.endDate) do
                local xshift = 0
                if usedDates[tostring(ele.date)] ~= nil then
                    xshift = math.ceil(usedDates[tostring(ele.date)] / 2)
                    if usedDates[tostring(ele.date)] % 2 == 0 then
                        xshift = -xshift
                    end
                    usedDates[tostring(ele.date)] = usedDates[tostring(ele.date)] + 1
                else
                    usedDates[tostring(ele.date)] = 1
                end
                tex.print(string.format([[\node[xshift=%d mm, fill opacity=.5,fill=red,circle,text width=3ex,inner sep=0mm, %s] at (%04d-%02d-%04d-%02d-%02d) {};]],
                    xshift, ele.tikz, ele.date:getyear(), ele.date:getmonth(), ele.date:getyear(), ele.date:getmonth(), ele.date:getday()))
                if ele.period == nil then break end
                ele.date:adddays(ele.period)
            end
        end
        tex.print([[\end{scope}]])
    tex.print([[\end{tikzpicture}]])
end
%    \end{macrocode}
% \end{macro}
% Prepare the module for exporting (only the functions that should be public)
%    \begin{macrocode}

semesterplannerLuaCal = {
    init = init,
    addAppointment = addAppointment,
    addDeadline = addDeadline,
    addExam = addExam,
    drawCalendar = drawCalendar,
}
return semesterplannerLuaCal
%</luaApp>
%    \end{macrocode}
%\Finale
