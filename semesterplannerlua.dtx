% \iffalse meta-comment
%
% Copyright (C) 2024 by Lukas Heindl <oss.heindl+latex@protonmail.com>
% ---------------------------------------------------------------------------
% This work may be distributed and/or modified under the
% conditions of the LaTeX Project Public License, either version 1.3
% of this license or (at your option) any later version.
% The latest version of this license is in
%   http://www.latex-project.org/lppl.txt
% and version 1.3 or later is part of all distributions of LaTeX
% version 2005/12/01 or later.
%
% This work has the LPPL maintenance status `maintained'.
%
% The Current Maintainer of this work is Lukas Heindl.
%
% This work consists of the files beamercolorthemeCatppuccin.dtx and beamercolorthemeCatppuccin.ins
% and the derived filebase beamercolorthemeCatppuccin.sty.
%
% \fi
%
% \iffalse
%<*driver>
\ProvidesFile{semesterplannerlua.dtx}
%</driver>
%<package>\NeedsTeXFormat{LaTeX2e}[1999/12/01]
%<package>\ProvidesPackage{semesterplannerlua}
%<*package>
   [2024/05/19 v1.1.1 Semesterplanner package in lua with tikz only]
%</package>
%
%<*driver>
\documentclass{ltxdoc}
\usepackage[a4paper,margin=25mm,left=50mm,nohead]{geometry}
\usepackage[numbered]{hypdoc}
\usepackage{array}
\usepackage{semesterplannerlua}[2024/02/27]
\EnableCrossrefs
\CodelineIndex
\RecordChanges
\begin{document}
  \DocInput{\jobname.dtx}
  \PrintChanges
  \PrintIndex
\end{document}
%</driver>
% \fi
%
% \CheckSum{362}
%
% \CharacterTable
%  {Upper-case    \A\B\C\D\E\F\G\H\I\J\K\L\M\N\O\P\Q\R\S\T\U\V\W\X\Y\Z
%   Lower-case    \a\b\c\d\e\f\g\h\i\j\k\l\m\n\o\p\q\r\s\t\u\v\w\x\y\z
%   Digits        \0\1\2\3\4\5\6\7\8\9
%   Exclamation   \!     Double quote  \"     Hash (number) \#
%   Dollar        \$     Percent       \%     Ampersand     \&
%   Acute accent  \'     Left paren    \(     Right paren   \)
%   Asterisk      \*     Plus          \+     Comma         \,
%   Minus         \-     Point         \.     Solidus       \/
%   Colon         \:     Semicolon     \;     Less than     \<
%   Equals        \=     Greater than  \>     Question mark \?
%   Commercial at \@     Left bracket  \[     Backslash     \\
%   Right bracket \]     Circumflex    \^     Underscore    \_
%   Grave accent  \`     Left brace    \{     Vertical bar  \|
%   Right brace   \}     Tilde         \~}
%
%
% \changes{v0.0.0}{2021/10/07}{First draft}
% \changes{v0.0.1}{2022/04/28}{Added new options (providing day
% representation in source code)}
% \changes{v0.0.2}{2022/04/30}{Write out timetable data to be able to build
% e.g. something with dmenu over it and provide an example dmenu lua
% script}
% \changes{v0.0.3}{2024/02/25}{Rename to semesterplannerLua and rewrite for
% l3build}
% \changes{v1.0.0}{2024/02/27}{First full release}
% \changes{v1.1.0}{2024/05/19}{Make ready for CTAN}
% \changes{v1.1.1}{2024/05/19}{Make ready for CTAN (2)}
%
%
% \DoNotIndex{\newcommand,\newenvironment,\\,\begin,\end,\def,\definecolor,\directlua,\endinput,\faBullhorn,\faCamera,\faFlag,\faQuestion,\faTimesCircle,\faWarning,\faWindows,\faYoutubePlay,\node,\pgfkeys,\pgfkeysvalueof,\protected,\raggedright,\textbf,\textcolor,\textwidth,\unexpanded,\",\x,\y,\year,\week,\tiny,\tikz@lib@cal@xshift,\tikz@lib@cal@yshift,\tikz,\textit,\t,\section,\phantom,\pgftransformyshift,\pgftransformxshif,\pgfmathsetlength,\pgf@ya,\pgf@xa,\openout,\n,\newwrite,\month,\minuteToFrac,\matrix,\l,\jobname,\init,\immediate,\ignorespaces,\ifdate,\hfil,\href,\foreach,\faStickyNoteO,\faPencil,\faComment,\faClockO,\faCalendar,\faBold,\draw,\d,\day,\checkKeys,\%,\RequirePackage,\usepackage,\usetikzlibrary,\pgftransformxshift}
%
% \providecommand*{\url}{\texttt}
% \GetFileInfo{semesterplannerlua.dtx}
% \title{\textsf{semesterplannerLua} --- Semesterplanner package in lua
%   with tikz only
% }
% \author{Lukas Heindl \\ \url{oss.heindl+latex@protonmail.com} \\ \faGithub:
% \url{https://github.com/atticus-sullivan/semesterplannerLua}}
% \date{\fileversion~from \filedate}
%
% \maketitle
%
% \begin{abstract}
% This package provides a mean to easily print a timetable e.g. for a
% semesterplan. The reason for this package to exist is that I wanted to
% reimplement \url{https://github.com/nlschn/semesterplanner/} with
% printing the timetable with |tikz| only (which is more easily to be
% modified) and with the ability of making entries spanning only a fraction
% of the column (for showing simultanious events).
% 
% Documents using this package need to be compiled with
% LuaLaTeX. The package requires |xcolor|, |fontawesome|, |tikz| (and |pgfkeys|).
% \end{abstract}
%
% \tableofcontents
%
% \section{Usage}
% Load with |\usepackage{semesterplannerlua}|
% \subsection{timetable}
%
% \DescribeEnv{timetable} |\begin{timetable}[opts]...\end{timetable}|\\
% This is the core environment of this package. Within it you can use
% |\lecture|, |\seminar|, |\tutorial|, |\officehour| and |\meeting|. All
% these commands are only defined inside the |timetable| environment, and
% have the same structure. It typesets a timetable with the specified
% entries. If you have a look at the |.sty| file you'll see that these
% macros just take a general macro and set some default values before. Thus
% it is easy to define new ones (keep in mind that you'll need
% |\makeatletter|) to access the general macro.
%
% Writes out a file containing the data from the timetable (your events) to
% be able to have just the |.tex| file containing the timetable, but do
% some scriping on the data (e.g. show the last, current and next events).
% For simplicity the data is stored in lua-syntax (|return|s a table of the
% events). This way we don't need to worry about any string escaping (like
% with csv) and no complicated parsers are needed. Be aware though that a
% simple |events = dofile"path"| has risks since |"path"| might contain lua
% code which is simply executed that way.
%
% |opts| are optional arguments (enclosed with |{}|) separated with |,|:
% \begin{description}
%     \item[|days|] List of the names of the days that should be set as
%     column names. Note that if you specify only 4 names only these 4
%     columns will be printed (with the first day being identified as
%     Monday)
%     \textit{Default: |Mon,Thue,Wend,Thur,Fri|}
%     \item[|dayse|] List of strings used in the code to idenfity the day.
%     This list is kind of a mapping string $\to$ integer.
%     \item[|start time|] Explicit start-time of the timetable given in minutes
%     (|HH*60 + MM|). Can be set as |start time/.evaluated={HH*60 + MM}|.
%     If this is empty, the start time is derived from the given events.
%     \textit{Default: |""|}
%     \item[|end time|] Equivalent to |start-time|
%     \textit{Default: |""|}
%     \item[|width|] Give the width of the timetable. (can be given e.g. as
%     |\textwidth| as this is directly given to tikz).
%     \textit{Default:} |\textwidth|
%     \item[|length|] Give the length of the timetable (measured in |cm|)
%     (has to be a straight number since this is needed in calculation)
%     \textit{Default: 10}
% \end{description}
% 
% \DescribeMacro{\lecture}    |\lecture   [opts]|
%
% \DescribeMacro{\tutorial}   |\tutorial  [opts]|
%
% \DescribeMacro{\seminar}    |\seminar   [opts]|
%
% \DescribeMacro{\officehour} |\officehour[opts]|
%
% \DescribeMacro{\meeting}    |\meeting   [opts]|
%
% |opts| are optional arguments (enclosed with |{}|) separated with |,|:
% \begin{description}
%     \item[|title|] Give the name of the lecture
%     \item[|speaker|] Give the name of the lecturer
%     \item[|location|] Give the place of the event (most probably the room or
%     an online plattform, see \ref{icons}). If you want to use
%     |\href{url}{string repr}|, keep in mind that you need to
%     |\unescaped{}| it, since all input to lua has to be text
%     \item[|day|] The weekday on which the event takes place. Has to be
%     one of those you specified in |dayse| (or by default: |M|,|T|,|W|,|Th|,|F|)
%     \item[|time|] The timespan of the event formatted as |HH:MM-HH:MM|
%     (24H clock)
%     \item[|prio|] The priority of the event (see \ref{icons})
%     \item[|scale width|] Specify the width of the entry in fractions
%     of column (use |0.5| to span half the column).
%     \item[|offset|] Shift the entry to the right. Specified in fractions
%     of column.
%     \item[|password|] Original purpose: Only included in the file written
%     out with the timetable data, to be able to make a script which
%     quickly copies url (from |location|) and |password| to the clipboard.
%     \item[|type|] Only included in the file written out with the
%     timetable data as well (Original purpose: To be able to exclude some
%     types)
%     \item[|tikz|] Free customizable event code. See the documentation
%     at the end for keys that can be used here (all keys in |/event|). To
%     simply pass arguments to the tikz-node that is being created for the
%     event use |tikz/.append={your arguments}| (be careful with |text width|,
%     |text height|, |text depth| as these keys are being used for
%     the dimensions of the node as well as with |anchor|)
%     \item[|content|] Is usually set automatically based on the other keys
%     (|title|for instance). This key can override this.
%     \item[|textcolor|] Usually set by the type of macro used
%     \item[|formatter|] internal stub for providing different formatting
%     options. Formatting functions need to be specified in the
%     |-timetable.lua| file
% \end{description}
% The entries |Day| and |Time| are mandatory since they are needed for the
% positioning of the node. All others are merely necessary for the content
% of the node and are therefore nor mandatory.
%
% \subsubsection{Special Notes}
% Note that the |length| argument does specify the length of the timetable
% without taking account of the column headers.
%
% Same goes for the |width| parameter regarding the labels containing the
% time on the right. Since in this case any tex-lenght is allowed, you can
% simply try to subtract the length of the clock label using something like
% |\settowidth{\length}{12:30}| to set a length to the length of a clock
% label and then subtract this from the length you want to specify.
%
% \hangindent=4.5em \hangafter=1 \textbf{Hint:} The content of the
% environment isn't processed by this package. Only the event commands (so
% to speak |\lecture|,|\tutorial|,|\seminar|,|\officehour|,|\meeting| are
% relevant.  All other contents are set immediately before the timetable.
% Therefore, if you wan to add e.g. a |\hspace*{10cm}| to shift the
% timetable to the left, the last line of the env would be the place to do
% so (there musn't be an empty line below since otherwise a new paragraph
% is started).
%
% \newpage
% \subsubsection{Example}
% |\begin{timetable}[|\\
% |        days={Mon,Thue,Wend,Thur,Fri},|\\
% |        start time/.evaluated={11*60}, end time/.evaluated={15*60}|\\
% |    ]|\\
% |    \lecture[title={TestingLectureLongOne},speaker={Doe},|\\
% |        location={RN1},day={W},time={12:30-13:30}]|\\
% |    \lecture[title={TestingLectureLongOne},speaker={John},|\\
% |        location={RN1},day={Th},time={12:30-13:30},offset=0.5,scale width=0.5]|\\
% |    \lecture[title={TestingLectureLongOne},speaker={Dr. Doe},|\\
% |        location={\zoom},day={T},time={12:30-13:30},prio={\phigh}]|\\
% |\end{timetable}|\\
%
% \begin{timetable}[days={Mon,Thue,Wend,Thur,Fri}, start
% time/.evaluated={11*60}, end time/.evaluated={15*60}]
%   \lecture[title={TestingLectureLongOne}, speaker={Heindl}, location={RN1},   day={W},  time={12:30-13:30}]
%   \lecture[title={TestingLectureLongOne}, speaker={Heindl}, location={RN1},   day={Th}, time={12:30-13:30}, offset=0.5, scale width=0.5]
%   \lecture[title={TestingLectureLongOne}, speaker={Heindl}, location={\zoom}, day={T},  time={12:30-13:30}, prio={\phigh}]
%   \hspace*{-.2\textwidth}
% \end{timetable}
%
% \subsection{Calendar related things}
% Now some environments and commands come which in generally are being used
% to typeset a table with the detailed information, but which also store
% the gathered data internally.\\
% This saved data can be used to output a calendar where the events (from
% the saved data) are highlighted with a mark (already passed days are
% crossed and the current day is highlighted as well).
%
% The different environments described below are similar in general, but
% differ in detail (mostly the list of argument that are passed and printed
% in the table).
%
% \newpage
% \paragraph{Example}\mbox{}\\
% |\begin{appointments}{true}|\\
% |    \appointment[print=true,shift=false,tikz={fill=black,rectangle},|\\
% |        date={2022-04-15}, course={Appointment with my dog},|\\
% |        end={2022-04-18}, period={1}]|\\
% |    \appointment[date={2022-04-14}, course={Appointment with my dog},|\\
% |        room={at home}, time={12:00}, prio={\pmandatory}]|\\
% |\end{appointments}|\\
% |\begin{exams}{false}|\\
% |    \exam[tikz={fill=yellow}, date={2022-06-20}, time={arround midday},|\\
% |        course={Driving}, desc={Mid-Term}, prio={\phigh},|\\
% |        room={University}, type={\oral}]|\\
% |\end{exams}|\\
% |\begin{deadlines}{false}|\\
% |    \deadline[tikz={fill=blue}, date={2022-06-22},|\\
% |        course={Submitting Driver-license form}, prio={\phigh}]|\\
% |\end{deadlines}|\\
% ||\\
% |\printSpCalendar[2]{2022-04-01}{2022-07-31}|

% \begin{appointments}{true}
%     \appointment[print=true,shift=false,tikz={fill=black,rectangle}, date={2022-04-15}, course={Appointment with my dog}, end={2022-04-18}, period={1}]
%     \appointment[date={2022-04-14}, course={Appointment with my dog}, room={at home}, time={12:00}, prio={\pmandatory}]
% \end{appointments}
% \begin{exams}{false} 
%     \exam[tikz={fill=yellow}, date={2022-06-20}, time={arround midday}, course={Driving}, desc={Mid-Term}, prio={\phigh}, room={University}, type={\oral}]
% \end{exams}
% \begin{deadlines}{false}
%     \deadline[tikz={fill=blue}, date={2022-06-22}, course={Submitting Driver-license form}, prio={\phigh}]
% \end{deadlines}
%
% \printSpCalendar[2]{2022-04-01}{2022-07-31}
%
% \DescribeMacro{\printSpCalendar} |\printSpCalendar[cols]{start_date}{end_date}|\\
% This is the macro to output such a calendar populated with the gathered
% events.\\
% Use the |cols| argument to specify the amount of columns being used
% (months set side by side).\\
% |start_date| and |end_date| both have to be specified in the |YYYY-MM-DD|
% format.
% 
% \subsubsection{Appointments}
% \DescribeEnv{appointments} |\begin{appointments}[Room]{true/false}...\end{appointments}|\\
% Environment to typeset a table of appointments. Stores the data
% internally to be able to typeset a calendar (with |\printSpCalendar|)
% with a mark for each event.\\
% The optional argument is used as header for the |Room| column (maybe for
% some appointments another title is more suitable).\\
% The mandatory argument which is |true/false| decides if the internally
% stored data used to typeset the calendar is being erased before this
% environment.
%
% \DescribeMacro{\appointment} |\appointment[opts]|\\
% |opts| are optional arguments (enclosed with |{}|) separated with |,|:
% \begin{description}
%     \item[|date|] Date of the appointment formatted as |YYYY-MM-DD|
%     \item[|course|] Name of the course
%     \item[|room|] Room (only shown in the table)
%     \item[|time|] Time of the event (free text independant of any
%     formatting, only shown in the table)
%     \item[|prio|] priority of the event (shown in the table) (see
%     \ref{icons})
%     \item[|desc|] Description of the event (shown in the table)
%     \item[|end|] upper boundary when the event should stop to repeat (with
%     |period|). Has to be set to make event periodic/repeating.
%     \item[|period|] Set the period of the repeating event.
%     \item[|draw|] |true/false| wether to draw a mark in the calendar
%     \item[|print|] |true/false| wether to display te entry in the
%     appointments table.
%     \item[|shift|] |true/false| wether to shift the mark in the calendar if
%     there is already another event on that date (to be able to see both
%     marks)
%     \item[|tikz|] Free customizable tikz code used for drawing the mark
%     in the calendar. See the documentation at the end for keys that can
%     be used here (all keys in |/cal|). To simply pass arguments to the
%     tikz-node that is being created for the event use |tikz/.append={your arguments}|\\
%     Use e.g. to set the color of the mark via |tikz/.append={green}|
% \end{description}
% \subsubsection{Exams}
% \DescribeEnv{exams} |\begin{exams}{true/false}...\end{exams}|\\
% Environment to typeset a table of exams. Stores the data
% internally to be able to typeset a calendar (with |\printSpCalendar|)
% with a mark for each event.\\
% The mandatory argument which is |true/false| decides if the internally
% stored data used to typeset the calendar is being erased before this
% environment.
%
% \DescribeMacro{\exam} |\exam[opts]|\\
% |opts| are optional arguments (enclosed with |{}|) separated with |,|:
% \begin{description}
%     \item[|date|] Date of the exam formatted as |YYYY-MM-DD|
%     \item[|course|] Name of the course
%     \item[|room|] Room (only shown in the table)
%     \item[|type|] Type of the exam (see \ref{icons}) (intended:
%     written/oral)
%     \item[|time|] Time of the event (free text independant of any
%     formatting, only shown in the table)
%     \item[|desc|] Description of the event (shown in the table)
%     \item[|draw|] |true/false| wether to draw a mark in the calendar
%     \item[|print|] |true/false| wether to display te entry in the
%     exams table.
%     \item[|shift|] |true/false| wether to shift the mark in the calendar if
%     there is already another event on that date (to be able to see both
%     marks)
%     \item[|tikz|] Free customizable tikz code used for drawing the mark
%     in the calendar. See the documentation at the end for keys that can
%     be used here (all keys in |/cal|). To simply pass arguments to the
%     tikz-node that is being created for the event use |tikz/.append={your arguments}|\\
%     Use e.g. to set the color of the mark via |tikz/.append={green}|
% \end{description}
%
% \subsubsection{Deadlines}
% \DescribeEnv{deadlines} |\begin{deadlines}{true/false}...\end{deadlines}|\\
% Environment to typeset a table of deadlines. Stores the data
% internally to be able to typeset a calendar (with |\printSpCalendar|)
% with a mark for each event.\\
% The mandatory argument which is |true/false| decides if the internally
% stored data used to typeset the calendar is being erased before this
% environment.
%
% \DescribeMacro{\deadline} |\deadline[opts]|\\
% |opts| are optional arguments (enclosed with |{}|) separated with |,|:
% \begin{description}
%     \item[|date|] Date of the deadline formatted as |YYYY-MM-DD|
%     \item[|course|] Name of the course
%     \item[|prio|] priority of the event (shown in the table) (see
%     \ref{icons})
%     \item[|desc|] Description of the event (shown in the table)
%     \item[|draw|] |true/false| wether to draw a mark in the calendar
%     \item[|print|] |true/false| wether to display te entry in the
%     deadlines table.
%     \item[|shift|] |true/false| wether to shift the mark in the calendar if
%     there is already another event on that date (to be able to see both
%     marks)
%     \item[|tikz|] Free customizable tikz code used for drawing the mark
%     in the calendar. See the documentation at the end for keys that can
%     be used here (all keys in |/cal|). To simply pass arguments to the
%     tikz-node that is being created for the event use |tikz/.append={your arguments}|\\
%     Use e.g. to set the color of the mark via |tikz/.append={green}|
% \end{description}
%
% \subsection{Icons}\label{icons}
% This package defines some modified fontawesome icons (they are being
% encircled with a white circle for better readability).\\
% \begin{tabular}{rl|rl}
%     |\zoom| & \textcolor{DodgerBlue}{\faCamera} &
%     |\teams| & \textcolor{DodgerBlue}{\faWindows}
%     \\
%     |\BBB| & \textcolor{DodgerBlue}{\faBold} &
%     |\youtube| & \textcolor{DodgerBlue}{\faYoutubePlay}
%     \\\hline
%     |\pmandatory| & \textcolor{red}{\faWarning} &
%     |\phigh| & \textcolor{red}{\faFlag}
%     \\
%     |\pmid| & \textcolor{yellow}{\faFlag} &
%     |\plow| & \textcolor{green}{\faFlag}
%     \\
%     |\pnone| & \textcolor{gray}{\faTimesCircle}
%     \\ \hline
%     |\tbd| & \faQuestion &
%     |\tba| & \faBullhorn
% \end{tabular}
%
%
%
% \newpage
% \section{Implementation}
% This package uses |semesterplannerLua| as prefix/directory where
% possible. Since this is not possible for latex macro names, in this
% occasions |semesterplannerLua@| is used as prefix.
% \subsection{semesterplannerlua.sty}
% \subsubsection{Global Stuff}
%    \begin{macrocode}
%<*package>
\RequirePackage{tikz}
\usetikzlibrary{calendar, positioning, shapes.misc, backgrounds}
\RequirePackage{pgfkeys}
\RequirePackage{xcolor}
\RequirePackage{fontawesome}
\RequirePackage{luapackageloader} % use the default lua path as well
%    \end{macrocode}
% Define some colors for the course types (can be globally overwritten)
%    \begin{macrocode}
\definecolor{seminar}{rgb}{1.0, 0.8, 0.0}
\definecolor{lecture}{rgb}{0.2, 0.7, 1.0}
\definecolor{tutorial}{rgb}{0.0, 0.8, 0.0}
\definecolor{meeting}{rgb}{0.8, 0.0, 0.0}
\definecolor{officehour}{rgb}{0.0, 0.4, 0.6}
\definecolor{DodgerBlue}{HTML}{1E90FF}
%    \end{macrocode}
% \begin{macro}{\semesterplannerLua@encircle}
% This macro puts a circle arround its argument for better readability. In
% this package this is used for the fontawesome symbols.
%    \begin{macrocode}
    \newcommand*{\semesterplannerLua@encircle}[1]{
        \begin{minipage}[b][1em][c]{1.5em}
            \begin{tikzpicture}
                \node[fill,circle,inner sep=1pt, color = white] {#1};
            \end{tikzpicture}   
        \end{minipage}
    }
%    \end{macrocode}
% \end{macro}
%    Commands for exams
%   \begin{macro}{\oral}
%    \begin{macrocode}
\protected\def\oral{\faComment}
%    \end{macrocode}
%   \end{macro}
%   \begin{macro}{\written}
%    \begin{macrocode}
\protected\def\written{\faPencil}
%    \end{macrocode}
%   \end{macro}
%   Commands for symbols of priority
%   \begin{macro}{\pmandatory}
%    \begin{macrocode}
    \protected\def\pmandatory{\semesterplannerLua@encircle{\textcolor{red}{\faWarning}}}
%    \end{macrocode}
%   \end{macro}
%   \begin{macro}{\phigh}
%    \begin{macrocode}
    \protected\def\phigh{\semesterplannerLua@encircle{\textcolor{red}{\faFlag}}}
%    \end{macrocode}
%   \end{macro}
%   \begin{macro}{\pmid}
%    \begin{macrocode}
    \protected\def\pmid{\semesterplannerLua@encircle{\textcolor{yellow}{\faFlag}}}
%    \end{macrocode}
%   \end{macro}
%   \begin{macro}{\plow}
%    \begin{macrocode}
    \protected\def\plow{\semesterplannerLua@encircle{\textcolor{green}{\faFlag}}}
%    \end{macrocode}
%   \end{macro}
%   \begin{macro}{\pnone}
%    \begin{macrocode}
    \protected\def\pnone{\semesterplannerLua@encircle{\textcolor{gray}{\faTimesCircle}}}
%    \end{macrocode}
%   \end{macro}

%   Commands for online platforms.
%   \begin{macro}{\teams}
%    \begin{macrocode}
    \protected\def\teams{\semesterplannerLua@encircle{\textcolor{DodgerBlue}{\faWindows}}}
%    \end{macrocode}
%   \end{macro}
%   \begin{macro}{\zoom}
%    \begin{macrocode}
    \protected\def\zoom{\semesterplannerLua@encircle{\textcolor{DodgerBlue}{\faCamera}}}
%    \end{macrocode}
%   \end{macro}
%   \begin{macro}{\youtube}
%    \begin{macrocode}
    \protected\def\youtube{\semesterplannerLua@encircle{\textcolor{red}{\faYoutubePlay}}}
%    \end{macrocode}
%   \end{macro}
%   \begin{macro}{\BBB}
%    \begin{macrocode}
    \protected\def\BBB{\semesterplannerLua@encircle{\textcolor{DodgerBlue}{\faBold}}}
%    \end{macrocode}
%   \end{macro}

%   Command for "To be determined" and "To be Announced"
%   \begin{macro}{\tbd}
%    \begin{macrocode}
    \protected\def\tbd{\faQuestion}
%    \end{macrocode}
%   \end{macro}
%   \begin{macro}{\tba}
%    \begin{macrocode}
    \protected\def\tba{\faBullhorn}
%    \end{macrocode}
%   \end{macro}


% Load the lua modules
%    \begin{macrocode}
\directlua{sp = require("semesterplannerLua_timetable.lua")}
\directlua{cal = require("semesterplannerLua_calendar.lua")}
%    \end{macrocode}
% Set all the pgfkeys required for the arguments. To achieve that the
% defaults are restored every time the environment is used, this is
% inside the environment definition. This of course disables all
% possibilities of setting a global default but enables setting local
% defaults for the events
%    \begin{macrocode}
    \pgfkeys{
%    \end{macrocode}
% |/semesterplannerLua| will be the pgf-path used for this package. Here
% all used keys are set (and initialized with defaults. |timetable/env/|:
% \begin{description}
%   \item[|days|] is a list of
%       strings representing the header names for the day columns in the
%       timetable (adding Sat and Sun (additional entries) will result in
%       two more columns.
%   \item[|start time|] can be used to set a fixed time where the timetable
%       starts (otherwise this is calculated from the entries) to enable
%       this behaviour this key has to be set to |HH*60 + MM| (easy way
%       is by using |start time/.evaluated={HH*60+MM}|)
%   \item[|end time|] equivalent to |start time|
%   \item[|width|] is
%       the horizontal width of the timetable (not including the column
%       headers on the top) this can be a latex length string or
%       |\textwidth| as well.
%   \item[|length|] is the vertical length of the
%       timetable (not including the clock labels on the side)
%       measured in cm (in future versions this may become measured in
%       pts for better interaction with the LaTeX lengths.
% \end{description}
%    \begin{macrocode}
        /semesterplannerLua/timetable/env/.cd,
        days/.initial={Mon,Thue,Wend,Thur,Fri}, days/.default={Mon,Thue,Wend,Thur,Fri},
        dayse/.initial={M,T,W,Th,F}, dayse/.default={M,T,W,Th,F},
        %
        start time/.initial=, start time/.default=,
        end time/.initial=, end time/.default=,
        %
        width/.initial=\textwidth, width/.default=\textwidth,
        length/.initial=10, length/.default=10,
        %
%    \end{macrocode}
% |timetable/event/|:
% \begin{description}
%   \item[|content|] is the content of the event (is passed on without any
%           formatting). Since this is passed to lua without modification
%           its value must be an unexpanded string (lua will simply print
%           it so the eventually the string will be evaluated)
%   \item[|time|] is a |HH:MM-HH:MM| string representing start- and
%       end-time of the event. Used in constructing the content as well
%   \item[|day|] is either |M|,|T|,|W|,|Th| or |F| specifying the day on
%       which the event takes place
%   \item[|tikz|] this key allows the user to manually pass options to the
%       node created for this event
%   \item[|scale width|] allows to scale the width of the event to be able
%       to draw overlapping events besides each other. Will usually be a
%       value between |0| and |1|.
%   \item[|offset|] same goal like |scale width| but shifts the event node
%       by the given value to the right. (Given as value between |0| and
%       |1| indicating how many columns the event should be shifted)
%   \item[|textcolor|] foreground color of the content text
%   \item[|title|] title (set in bold by default)
%   \item[|speaker|] 
%   \item[|location|] 
%   \item[|prio|] 
%   \item[|formatter|] this is special
% \end{description}
%    \begin{macrocode}
        /semesterplannerLua/timetable/event/.cd,
        % event arguments
        content/.initial=, content/.default=,
        %
        time/.initial=, time/.default=,
        day/.initial=, day/.default=,
        %
        tikz/.initial=, tikz/.default=,
        scale width/.initial=1, scale width/.default=1,
        offset/.initial=0, offset/.default=0,
        %
        textcolor/.initial=, textcolor/.default=,
        title/.initial=, title/.default=,
        speaker/.initial=, speaker/.default=,
        location/.initial=, location/.default=,
        password/.initial=, password/.default=,
        prio/.initial=, prio/.default=,
        type/.initial=, type/.default=,
        formatter/.initial=timetableformatter, formatter/.default=timetableformatter,
        %
%    \end{macrocode}
% |calendar/|:
% \begin{description}
%   \item[|draw|] 
%   \item[|room|] 
%   \item[|prio|] 
%   \item[|course|] 
%   \item[|desc|] 
%   \item[|start|] 
%   \item[|end|] 
%   \item[|tikz|] 
%   \item[|period|] 
%   \item[|shift|] 
%   \item[|print|] Only makes sence if the command is suffixed by a |%|
%       otherwise somehow a space gets inserted (eventhough the |%| is
%       inserted from lua as well
% \end{description}
%    \begin{macrocode}
        /semesterplannerLua/calendar/.cd,
        draw/.initial={true}, draw/.default={true},
        room/.initial={}, room/.default={},
        time/.initial={}, time/.default={},
        prio/.initial={}, prio/.default={},
        course/.initial={}, course/.default={},
        desc/.initial={}, desc/.default={},
        type/.initial={}, type/.default={},
        date/.initial={}, date/.default={},
        end/.initial={}, end/.default={},
        tikz/.initial={}, tikz/.default={},
        period/.initial={nil}, period/.default={nil},
        shift/.initial={true}, shift/.default={true},
        print/.initial={true}, print/.default={true},
    }
%    \end{macrocode}
%
% \subsection{Tikz Calendar add weekday labels}
%    \begin{macrocode}
\tikzoption{day headings}{\tikzstyle{day heading}=[#1]}
\tikzstyle{day heading}=[]
\tikzstyle{day letter headings}=[
    execute before day scope={ \ifdate{day of month=1}{%
      \pgfmathsetlength{\pgf@ya}{\tikz@lib@cal@yshift}%
      \pgfmathsetlength\pgf@xa{\tikz@lib@cal@xshift}%
      \pgftransformyshift{-\pgf@ya}
      \foreach \d/\l in {0/M,1/T,2/W,3/T,4/F,5/S,6/S} {
        \pgf@xa=\d\pgf@xa%
        \pgftransformxshift{\pgf@xa}%
        \pgftransformyshift{\pgf@ya}%
        \node[every day,day heading]{\l};%
      } 
    }{}%
  }%
]
%    \end{macrocode}
%
% \subsubsection{Local Stuff (timetable-env local)}
% \begin{environment}{timetable}
% This is the environment doing all the stuff. To gate the positions where
% the corresponding macros can be used (and in terms of pgfkeys for reasons
% of default values) all the macros used are put into the environment.
%    \begin{macrocode}
\newenvironment{timetable}[1][]{
    \section*{\faClockO~Timetable}
%    \end{macrocode}
%
%   Read the argumens given by the user after restoring the defaults
%   (Restoring currently makes no sense, since they are created a few
%   lines above anyways, but creation might be moved outside the
%   environment some day.\\
%   Afterwards the lua module is beeing initialized (erase data from
%   possible previous runs.
%    \begin{macrocode}
    \pgfkeys{/semesterplannerLua/timetable/env/.cd, days,dayse, start time,end time, width,length, #1}
    \directlua{sp.init{
        days=[[\pgfkeysvalueof{/semesterplannerLua/timetable/env/days}]],
        min=[[\pgfkeysvalueof{/semesterplannerLua/timetable/env/start time}]],
        max=[[\pgfkeysvalueof{/semesterplannerLua/timetable/env/end time}]],
        dayse=[[\pgfkeysvalueof{/semesterplannerLua/timetable/env/dayse}]]}}
%    \end{macrocode}
%   \begin{macro}{\semesterplannerLua@event}
% Is used to pass the event to the lua engine which in turn will collect
% the event to draw it in the end. For that the arguments given are parsed
% after restoring the pgf keys to their default values. The optional
% argument herby is a sequence of pgf keys, the second argument is
% a string representing the content (this MUST be unexpanded since
% this is passed to lua which in turn will pass it unmodified back)
%    \begin{macrocode}
    \newcommand{\semesterplannerLua@event}[1][]{
        \pgfkeys{/semesterplannerLua/timetable/event/.cd,content,time,day,tikz,scale
        width,offset,textcolor,title,speaker,type,location,password,prio,formatter, ##1}
        \directlua{
            sp.addEvent{
                time="\pgfkeysvalueof{/semesterplannerLua/timetable/event/time}",
                day="\pgfkeysvalueof{/semesterplannerLua/timetable/event/day}",
                tikz=[[\pgfkeysvalueof{/semesterplannerLua/timetable/event/tikz}]],
                offset=\pgfkeysvalueof{/semesterplannerLua/timetable/event/offset},
                scale_width=\pgfkeysvalueof{/semesterplannerLua/timetable/event/scale width},
                formatter=\pgfkeysvalueof{/semesterplannerLua/timetable/event/formatter},
                textcolor=[[\pgfkeysvalueof{/semesterplannerLua/timetable/event/textcolor}]],
                title=[[\pgfkeysvalueof{/semesterplannerLua/timetable/event/title}]],
                speaker=[[\pgfkeysvalueof{/semesterplannerLua/timetable/event/speaker}]],
                location=[[\pgfkeysvalueof{/semesterplannerLua/timetable/event/location}]],
                password=[[\pgfkeysvalueof{/semesterplannerLua/timetable/event/password}]],
                prio=[[\pgfkeysvalueof{/semesterplannerLua/timetable/event/prio}]],
                type=[[\pgfkeysvalueof{/semesterplannerLua/timetable/event/type}]],
            }
        }
    }
%    \end{macrocode}
%   \end{macro}
% Short-hand macros for different events using the corresponding
% background color
%
%   \begin{macro}{\lecture}
%    \begin{macrocode}
    \newcommand{\lecture}[1][]{
        \semesterplannerLua@event[tikz={fill=lecture,}, textcolor=white, type=lect, ##1]
        \ignorespaces
    }
%    \end{macrocode}
%   \end{macro}
%   \begin{macro}{\seminar}
%    \begin{macrocode}
    \newcommand{\seminar}[1][]{
        \semesterplannerLua@event[tikz={fill=seminar,}, textcolor=white, type=sem, ##1]
        \ignorespaces
    }
%    \end{macrocode}
%   \end{macro}
%   \begin{macro}{\tutorial}
%    \begin{macrocode}
    \newcommand{\tutorial}[1][]{
        \semesterplannerLua@event[tikz={fill=tutorial,}, textcolor=white, type=tut, ##1]
        \ignorespaces
    }
%    \end{macrocode}
%   \end{macro}
%   \begin{macro}{\meeting}
%    \begin{macrocode}
    \newcommand{\meeting}[1][]{
        \semesterplannerLua@event[tikz={fill=meeting,}, textcolor=white, type=meet, ##1]
        \ignorespaces
    }
%    \end{macrocode}
%   \end{macro}
%   \begin{macro}{\officehour}
%    \begin{macrocode}
    \newcommand{\officehour}[1][]{
        \semesterplannerLua@event[tikz={fill=officehour,}, textcolor=white, type=office, ##1]
        \ignorespaces
    }
%    \end{macrocode}
%   \end{macro}
%    \begin{macrocode}
}{
%    \end{macrocode}
% At the end of the environment after all events have been collected,
% generate and output the tikz code needed to draw the timetable.
%    \begin{macrocode}
    \directlua{sp.draw(
        [[\pgfkeysvalueof{/semesterplannerLua/timetable/env/length}]],
        [[\pgfkeysvalueof{/semesterplannerLua/timetable/env/width}]])}
}
%    \end{macrocode}
% \end{environment}
%
%    \begin{macrocode}

%    \end{macrocode}
% \begin{macro}{printSpCalendar}
% Print a calendar from startDate to endDate (encoded as YYYY-MM-DD) as
% one calendar per month in a matrix with the given amount of columns
%    \begin{macrocode}
\newcommand{\printSpCalendar}[3][3]{\directlua{cal.drawCalendar("#2", "#3", #1)}}
%    \end{macrocode}
% \end{macro}
%    \begin{macrocode}

\newenvironment{appointments}[2][Room]{
    \directlua{cal.init(#2)}
    \newcommand{\appointment}[1][]{%
        \pgfkeys{/semesterplannerLua/calendar/.cd,draw,room,time,prio,course,desc,date,end,tikz,period,type,shift,print, ##1}%
        \directlua{
            cal.addAppointment{draw=\pgfkeysvalueof{/semesterplannerLua/calendar/draw},
            room=[[\pgfkeysvalueof{/semesterplannerLua/calendar/room}]],
            time=[[\pgfkeysvalueof{/semesterplannerLua/calendar/time}]],
            prio=[[\pgfkeysvalueof{/semesterplannerLua/calendar/prio}]],
            course=[[\pgfkeysvalueof{/semesterplannerLua/calendar/course}]],
            desc=[[\pgfkeysvalueof{/semesterplannerLua/calendar/desc}]],
            date=[[\pgfkeysvalueof{/semesterplannerLua/calendar/date}]],
            endDate=[[\pgfkeysvalueof{/semesterplannerLua/calendar/end}]],
            tikz=[[\pgfkeysvalueof{/semesterplannerLua/calendar/tikz}]],
            period=\pgfkeysvalueof{/semesterplannerLua/calendar/period},
            shift=\pgfkeysvalueof{/semesterplannerLua/calendar/shift},
            print=\pgfkeysvalueof{/semesterplannerLua/calendar/print}}}%
        \ignorespaces
        }
    \section*{\faCalendar~Appointments}
    \begin{tabular}{rlllll}
        \textbf{Date}&\textbf{Time}&\textbf{Course}&\textbf{Description}&\textbf{#1}&\textbf{Prio.}\\
}{
    \end{tabular}
}

\newenvironment{exams}[1]{
    \directlua{cal.init(#1)}
    \newcommand{\exam}[1][]{%
        \pgfkeys{/semesterplannerLua/calendar/.cd,draw,room,time,prio,course,desc,date,end,tikz,period,type,shift,print, ##1}%
        \directlua{
            cal.addExam{
                draw=\pgfkeysvalueof{/semesterplannerLua/calendar/draw},
                room=[[\pgfkeysvalueof{/semesterplannerLua/calendar/room}]],
                time=[[\pgfkeysvalueof{/semesterplannerLua/calendar/time}]],
                course=[[\pgfkeysvalueof{/semesterplannerLua/calendar/course}]],
                desc=[[\pgfkeysvalueof{/semesterplannerLua/calendar/desc}]],
                date=[[\pgfkeysvalueof{/semesterplannerLua/calendar/date}]],
                tikz=[[\pgfkeysvalueof{/semesterplannerLua/calendar/tikz}]],
                type=[[\pgfkeysvalueof{/semesterplannerLua/calendar/type}]],
                shift=\pgfkeysvalueof{/semesterplannerLua/calendar/shift},
                print=\pgfkeysvalueof{/semesterplannerLua/calendar/print}}}%
        \ignorespaces
        }
    \section*{\faStickyNoteO~Exams}
    \begin{tabular}{rllll}
        \textbf{Date}&\textbf{Time}&\textbf{Course}&\textbf{Type}&\textbf{Note}\\
}{
    \end{tabular}
}

\newenvironment{deadlines}[1]{
    \directlua{cal.init(#1)}
    \newcommand{\deadline}[1][]{%
        \pgfkeys{/semesterplannerLua/calendar/.cd,draw,room,time,prio,course,desc,date,end,tikz,period,type,shift,print, ##1}%
        \directlua{
            cal.addDeadline{
                draw=\pgfkeysvalueof{/semesterplannerLua/calendar/draw},
                course=[[\pgfkeysvalueof{/semesterplannerLua/calendar/course}]],
                desc=[[\pgfkeysvalueof{/semesterplannerLua/calendar/desc}]],
                date=[[\pgfkeysvalueof{/semesterplannerLua/calendar/date}]],
                tikz=[[\pgfkeysvalueof{/semesterplannerLua/calendar/tikz}]],
                prio=[[\pgfkeysvalueof{/semesterplannerLua/calendar/prio}]],
                shift=\pgfkeysvalueof{/semesterplannerLua/calendar/shift},
                print=\pgfkeysvalueof{/semesterplannerLua/calendar/print}}}%
        \ignorespaces
        }
    \section*{\faStickyNoteO~Deadlines}
    \begin{tabular}{rlll}
        \textbf{Date}&\textbf{Course}&\textbf{Description}&\textbf{Prio}\\
}{
    \end{tabular}
}
%    \end{macrocode}
\endinput
%
% \iffalse
%</package>
% \fi
%
% \Finale
\endinput
